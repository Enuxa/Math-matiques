\documentclass[]{article}
\usepackage[utf8]{inputenc}
\usepackage{pdfpages}
\usepackage{amsmath}
\usepackage{amssymb}
\usepackage{graphicx}
\usepackage{geometry}
\usepackage{enumitem}
\usepackage{amsthm}

\geometry{hmargin=2cm}

% Environnement type théorème
\newtheorem{mythm}{Théorème}
\newtheorem{myproposition}{Proposition}
\newtheorem{myproperty}{Propriété}
\newtheorem{mylemma}{Lemme}
\newtheorem{mycor}{Corollaire}

% Environnement type texte
\theoremstyle{remark}
\newtheorem{mynot}{Notation}
\newtheorem{myrem}{Remarque}
\newtheorem{myexer}{Exercice}
\newtheorem{myproof}{Preuve}
\newtheorem{myexmpl}{Exemple}

% Environnement de définition
\theoremstyle{definition}
\newtheorem{mydef}{Définition}

\setlist[itemize]{label=-}

% Carré de fin de preuve
\newcommand{\cqfd}{
	\hfill$\square$
}

% "Checkmark" de fin d'étape de preuve
\newcommand{\checked}{
	\hfill$\checkmark$
}

% Définition de fonction
\newcommand{\func}[5]{
#1 ~ : ~ \left\{ \begin{array}{lcl}
	#2 & \longrightarrow & #3 \\
	#4 & \longmapsto & #5
\end{array}
\right.
}

\newcommand{\funcinline}[5]{
#1 ~ : ~ #2 \longrightarrow #3, ~ #4 \longmapsto #5
}

\newcommand{\funcshort}[3]{
#1 ~ : ~ #2 \longrightarrow #3
}

\newenvironment{proofpart}[1]{
	\noindent
	{\boldmath #1}
}{
	\checkmark
}

\begin{document}
\section*{Exercice 2}
	Soient $m, n > 0$, on cherche à déterminer $\mathbb{U}_m \cap \mathbb{U}_n$.
	
	Tout d'abord $\mathbb{U}_{m \land n} \subseteq \mathbb{U}_m$ et $\mathbb{U}_{m \land n} \subseteq \mathbb{U}_n$, alors $\mathbb{U}_{m \land n} \subseteq \mathbb{U}_m \cap \mathbb{U}_n$
	
	De plus, pour tout $z \in \mathbb{U}_m \cap \mathbb{U}_n$, on a que l'ordre de $z$ divise à la fois $m$ et $n$, donc il divise $m \land n$, et donc $z \in \mathbb{U}_{m \land n}$.
	
	Alors $\mathbb{U}_m \cap \mathbb{U}_n \subseteq \mathbb{U}_{m \land n}$.
	
	On a donc montré que $\mathbb{U}_{m \land n} = \mathbb{U}_m \cap \mathbb{U}_n$

Autre rédaction :

$$\mathbb{U}_n \cap \mathbb{U}_m = \{z \in \mathbb{C} ~ | ~ z^m = z^n = 1 \}$$

$$\mathbb{U}_n \cap \mathbb{U}_m = \{z \in \mathbb{U} \text{ d'ordre fini } d ~ | ~ d | m \text{ et } d | n\}$$

$$\mathbb{U}_n \cap \mathbb{U}_m = \{z \in \mathbb{U} \text{ d'ordre fini } d ~ | ~ d | (m \land n)\}$$

$$\mathbb{U}_n \cap \mathbb{U}_m = \{z \in \mathbb{U} ~ | ~ s^{m \land n}\}$$

$$\mathbb{U}_n \cap \mathbb{U}_m = \mathbb{U}_{m \land n}$$

\section*{Exercice 3}
	Soient $x$ et $y$ deux éléments d'un groupe $G$ tels que $x$ soit d'ordre 5 et $xyx^{-1}=y^2$.
	
	On a pour tout $n$ :
	
	$$x^n y x^{-1} = y^{2^n}$$
	
	en effet : $x y x^{-1} = y^2$ et si $x^n y x^{-n} = y^{2^n}$, alors :
	
	$$x^{n+1} y x^{-n-1} = x\left(x^n y x^{-n}\right)x^{-1}$$

	$$x^{n+1} y x^{-n-1} = xy^{2^n}x^{-1}$$

	$$x^{n+1} y x^{-n-1} = \left(xyx^{-1}\right)^{2^n}$$

	$$x^{n+1} y x^{-n-1} = \left(y^2\right)^{2^n}$$

	$$x^{n+1} y x^{-n-1} = y^{2 \cdot 2^n}$$

	$$x^{n+1} y x^{-n-1} = y^{2^{n+1}}$$

On a donc que $x^5 y x^{-5} = y^{32}$, or $x$ étant d'ordre 5, on a donc $y = y^{32}$, c'est-à-dire $y^{31} = e$.

31 est un nombre premier, on peut donc affirmer que l'ordre de $y$ est 31, car aucun diviseur strict $d$ de 31 ne vérifie $y^d = e$ (l'unique diviseur strict étant 1).

\section*{Exercice 4}

Soient $g$ et $h$ dans $G$ avec $(*) ~ ghg^{-1}=h^n$ pour un certain $n$ fixé.

Montrons par récurrence sur $k \in \mathbb{N}$ que $g^{\alpha_1} h^{\beta_1} ... g^{\alpha_k} h^{\beta_k}$ peut être écrit sous la forme $g^{\alpha} h^{\beta}$.

On utilisera le fait que pour tout $x, y \in G$ et pour tout $k \in \mathbb{Z}$, $(**) ~ \left(ghg^{-1}\right)^k = \underbrace{ghg^{-1} ghg^{-1} ... ghg^{-1}}_{k \text{ fois}} = g h^k g^{-1}$.

\underline{$k = 0$ :}

Une telle écriture désigne $e$

\underline{$k \geqslant 0$ :}

Supposons la proposition vraie jusqu'à un certain entier $k$, et on considère

$$g^{\alpha_1} h^{\beta_1} ... g^{\alpha_k} h^{\beta_k} g^{\alpha_{k+1}} h^{\beta_{k+1}}$$

Par hypothèse de récurrence :

$$g^{\alpha_1} h^{\beta_1} ... g^{\alpha_k} h^{\beta_k} g^{\alpha_{k+1}} h^{\beta_{k+1}} = g^\alpha h^\beta g^{\alpha_{k+1}} h^{\beta_{k+1}}$$

$$g^{\alpha_1} h^{\beta_1} ... g^{\alpha_k} h^{\beta_k} g^{\alpha_{k+1}} h^{\beta_{k+1}} = g^{\alpha + \alpha_{k+1}} g^{-\alpha_{k+1}} h^\beta g^{\alpha_{k+1}} h^{\beta_{k+1}}$$

$$g^{\alpha_1} h^{\beta_1} ... g^{\alpha_k} h^{\beta_k} g^{\alpha_{k+1}} h^{\beta_{k+1}} = g^{\alpha + \alpha_{k+1}} \left(g^{-\alpha_{k+1}} h^\beta g^{\alpha_{k+1}}\right) h^{\beta_{k+1}}$$

$$g^{\alpha_1} h^{\beta_1} ... g^{\alpha_k} h^{\beta_k} g^{\alpha_{k+1}} h^{\beta_{k+1}} = g^{\alpha + \alpha_{k+1}} \left(g^{-\alpha_{k+1}} h g^{\alpha_{k+1}}\right)^\beta h^{\beta_{k+1}}, ~ \text{d'après } (**)$$

$$g^{\alpha_1} h^{\beta_1} ... g^{\alpha_k} h^{\beta_k} g^{\alpha_{k+1}} h^{\beta_{k+1}} = g^{\alpha + \alpha_{k+1}} \left(g^{-\alpha_{k+1} + 1} h^n g^{\alpha_{k+1} - 1}\right)^\beta h^{\beta_{k+1}}, ~ \text{d'après } (*)$$

$$g^{\alpha_1} h^{\beta_1} ... g^{\alpha_k} h^{\beta_k} g^{\alpha_{k+1}} h^{\beta_{k+1}} = g^{\alpha + \alpha_{k+1}} \left(g^{-\alpha_{k+1} + 1} h g^{\alpha_{k+1} - 1}\right)^{n \cdot \beta} h^{\beta_{k+1}}, ~ \text{d'après } (**)$$

$$g^{\alpha_1} h^{\beta_1} ... g^{\alpha_k} h^{\beta_k} g^{\alpha_{k+1}} h^{\beta_{k+1}} = g^{\alpha + \alpha_{k+1}} \left(g^{-\alpha_{k+1} + 2} h^n g^{\alpha_{k+1} - 2}\right)^{n \cdot \beta} h^{\beta_{k+1}}, ~ \text{d'après } (*)$$

$$g^{\alpha_1} h^{\beta_1} ... g^{\alpha_k} h^{\beta_k} g^{\alpha_{k+1}} h^{\beta_{k+1}} = g^{\alpha + \alpha_{k+1}} \left(g^{-\alpha_{k+1} + 2} h g^{\alpha_{k+1} - 2}\right)^{n^2 \cdot \beta} h^{\beta_{k+1}}, ~ \text{d'après } (**)$$

$$...$$

$$g^{\alpha_1} h^{\beta_1} ... g^{\alpha_k} h^{\beta_k} g^{\alpha_{k+1}} h^{\beta_{k+1}} = g^{\alpha + \alpha_{k+1}} h^{n^{\alpha_{k+1}} \cdot \beta} h^{\beta_{k+1}}$$

$$g^{\alpha_1} h^{\beta_1} ... g^{\alpha_k} h^{\beta_k} g^{\alpha_{k+1}} h^{\beta_{k+1}} = g^{\alpha + \alpha_{k+1}} h^{n^{\alpha_{k+1}} \cdot \beta + \beta_{k+1}}$$

$$g^{\alpha_1} h^{\beta_1} ... g^{\alpha_k} h^{\beta_k} g^{\alpha_{k+1}} h^{\beta_{k+1}} = g^{\alpha'} h^{\beta'}$$

Avec $\alpha' = \alpha + \alpha_{k+1}$ et $\beta' = n^{\alpha_{k+1}} \cdot \beta + \beta_{k+1}$.

\section*{Exercice 5}

\begin{enumerate}
	\item On considère la relation d'équivalence $\sim$ définie par $x \sim y \Longleftrightarrow f(x) = f(y)$.
	
	On remarque immédiatement que :
	$$x \sim y \Longleftrightarrow f(x^{-1}y) = e$$
	$$x \sim y \Longleftrightarrow x^{-1}y \in \ker f$$
	$$x \sim y \Longleftrightarrow \left(x^{-1}y\right)\ker f = \ker f$$
	$$x \sim y \Longleftrightarrow y \ker f = x \ker f$$

	On a alors, par le théorème du passage au quotient, une injection $\funcshort{\varphi}{G/\ker f}{G'}$ définie par $\varphi(g \ker f) = f(g)$ qui de plus est surjective car $f$ l'est.
	
	On en déduit
	
	$$\text{Card } (G) = \text{Card } (G') \cdot \text{Card } (\ker f)$$
	
	\item On suppose qu'il existe un élément $g \in G$ tel que son ordre, noté $n$, est premier à celui de $\ker f$.
	
	On a toujours que l'ordre de $f(g)$ divise $n$.
	
	On sait également que $n$ est premier à l'ordre de $\ker f$, alors pour tout diviseur strict $k$ de $n$, $g^k$ est d'ordre divisant $n$ donc d'ordre toujours premier à celui de $\ker f$, d'où $g^k \notin \ker f$.
	
	Ainsi $e \neq f(g^k) = \left(f(g)\right)^k$, $n$ est donc l'ordre de $f(g)$.
\end{enumerate}

\section*{Exercice 6}

On note pour tout $P \in X$ et tout $g \in G$ $g \cdot P = \{g * x ~ | ~ x \in P\}$.

\begin{enumerate}
	\item Soient $P \in X$ et $g \in G$, on a que $g \cdot P$ est l'image de $P$ par la bijection $$\func{\varphi_g}{G}{G}{x}{g * x}$$ (de bijection réciproque $\varphi_{g^{-1}}$) et donc que $P$ et $g \cdot P$ sont équipotents, d'où $g \cdot P \in X$.
	
	\item Pour tout $g, h \in G$ et tout $P \in X$ on a :
	
	$$g \cdot \left(h \cdot P\right) = g \cdot \{h * x ~ | ~ x \in \}$$
	
	$$g \cdot \left(h \cdot P\right) = \{g * h * x ~ | ~ x \in \}$$
	
	$$g \cdot \left(h \cdot P\right) = \{(g * h) * x ~ | ~ x \in \}$$
	
	$$g \cdot \left(h \cdot P\right) = (g * h) \cdot \{x ~ | ~ x \in \}$$
	
	$$g \cdot \left(h \cdot P\right) = (g * h) \cdot P$$
	
	De plus $e \cdot P = \{e * x ~ | ~ x \in P\} = P$
	
	Ainsi, l'application $G \times X \longrightarrow X, ~ (g, P) \longmapsto g \cdot P$ est bien une action de groupe.
	
	\item Soit $O_1, ...O_n$ une énumération des orbites et pour tout $k=1, ...n$ on pose $P_k \in O_k$.
	
	On a alors :
	
	$$|X| = \sum_{k = 1}^{n} \text{Card } (G \cdot P_k) \equiv m \mod p$$

	$$\sum_{k = 1}^{n} \frac{\text{Card }(G)}{\text{Card }(\text{Stab}_G (P_k))}\equiv m \mod p$$

	$$\sum_{k = 1}^{n} \frac{p^\alpha m}{\text{Card }(\text{Stab}_G (P_k))}\equiv m \mod p$$

	Or $\text{Stab}_G (P_k)$ est un sous-groupe de $G$, donc son cardinal est soit de la forme $p^\beta$ soit $p^\beta m$, avec $\beta$ éventuellement nul, il existe donc nécessairement un indice $k$ tel que $\text{Card }(\text{Stab}_G(P_k))$ soit égal à $p^\alpha$ et on note $P_0 = P_k$.

	\item D'après l'équation aux classes :
	
	$$\text{Card } (G \cdot P_0) \cdot \text{Card }(\text{Stab}_G (P_0)) = \text{Card } (G) = p^\alpha m$$
	
	or $p$ ne divise pas $\text{Card } (G \cdot P_0)$, alors nécessairement $p^\alpha$ divise $\text{Card }(\text{Stab}_G (P_0))$, d'où $\text{Card }(\text{Stab}_G (P_0)) \geqslant p^\alpha$.
	
	\item Pour tout $g \in \text{Stab}_G (P_0)$, $g \cdot P_0 = \{g * y ~ | ~ y \in P_0\} = P_0$, en particulier $g*x \in P_0$, ainsi $\text{Stab}_G(P_0)x \subseteq P_0$.
	
	De plus, pour la même raison qu'en 1 (mais avec la bijection $g \mapsto g * x$), on a que $\text{Stab}_G(P_0) x$ est équipotent à $\text{Stab}_G(P_0)$. On a donc que $\text{Card } (\text{Stab}_G(P_0)) x \geqslant p^\alpha = \text{Card } (P_0)$, alors $\text{Stab}_G(P_0) x$ est maximal dans $P_0$ (au sens de l'inclusion).
	
	On a donc montré $\text{Stab}_G(P_0) x = P_0$.
	
	\item $\text{Stab}_G(P_0)$ est équipotent à $P_0$, donc $\text{Stab}_G(P_0)$ est un sous-groupe d'ordre $p^\alpha$.
\end{enumerate}

\end{document}
