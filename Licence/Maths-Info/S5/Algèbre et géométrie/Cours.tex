\documentclass[]{article}
\usepackage[utf8]{inputenc}
\usepackage{pdfpages}
\usepackage{amsmath}
\usepackage{amssymb}
\usepackage{graphicx}
\usepackage{geometry}
\usepackage{enumitem}
\usepackage{amsthm}
\usepackage{mathrsfs}
\usepackage[hidelinks]{hyperref}

\geometry{hmargin=2cm}

\title{Algèbre et géométrie 1}

\author{Patrick Le Meur et Pierre Gervais}

% Environnement type théorème
\newtheorem{mythm}{Théorème}
\newtheorem{myproposition}{Proposition}
\newtheorem{myproperty}{Propriété}
\newtheorem{mylemma}{Lemme}
\newtheorem{mycor}{Corollaire}

% Environnement type texte
\theoremstyle{remark}
\newtheorem{mynot}{Notation}
\newtheorem{myrem}{Remarque}
\newtheorem{myexer}{Exercice}
\newtheorem{myproof}{Preuve}
\newtheorem{myexmpl}{Exemple}

% Environnement de définition
\theoremstyle{definition}
\newtheorem{mydef}{Définition}

\setlist[itemize]{label=-}

% Carré de fin de preuve
\newcommand{\cqfd}{
	\hfill$\square$
}

% "Checkmark" de fin d'étape de preuve
\newcommand{\checked}{
	\hfill$\checkmark$
}

% Définition de fonction
\newcommand{\func}[5]{
#1 \, : \, \left\{ \begin{array}{lcl}
	#2 & \longrightarrow & #3 \\
	#4 & \longmapsto & #5
\end{array}
\right.
}

\newcommand{\funcinline}[5]{
#1 \, : \, #2 \longrightarrow #3, ~ #4 \longmapsto #5
}

\newcommand{\funcshort}[3]{
#1 \, : \, #2 \longrightarrow #3
}

\newcommand{\anonfunc}[4]{
\left\{ \begin{array}{lcl}
	#1 & \longrightarrow & #2 \\
	#3 & \longmapsto & #4
\end{array}
\right.
}

\newenvironment{proofpart}[1]{
	\noindent
	{\textbf{\boldmath #1}}
}{
	\checkmark
}

\newcommand*{\ClGa}[2]%
{\ensuremath{%
    #1/\!\raisebox{-.65ex}{\ensuremath{#2}}}}

\newcommand*{\ClDr}[2]%
{\ensuremath{%
    \raisebox{-.65ex}{\ensuremath{#2}}\backslash #1}}

\begin{document}

\maketitle

\tableofcontents

\part{Groupes}

\section{Définitions et premiers exemples}

\begin{mydef}
	Un \textit{groupe} est un couple $(G, *)$ où
	\begin{itemize}
		\item $G$ est un ensemble
		\item $\func{*}{G \times G}{G}{(g,h)}{g*h}$ est une loi de composition interne associative admettant un élément neutre $e$, c'est à dire tel que $\forall g \in G, g * e = e * g = g$
		\item tout élément $g$ admet un symétrique pour $*$ noté $g^{-1}$ tel que $g*g^{-1}=g^{-1}*g=e$
	\end{itemize}
\end{mydef}

\begin{myrem}
\leavevmode
\begin{itemize}
	\item L'élément neutre et le symétrique d'un élément donné est unique.
	\item Pour tout $g, h \in G$ on a $(g*h^{-1})=h^{-1}*g^{-1}$
	\item Si on a $gh=e$, alors $g=h^{-1}$
	\item Soit $g \in G$ et $n > 0$, on définit $g^n=\underbrace{g*g*g...g}_{n \text{ fois}}, ~ g^0=e, ~ g^{n+1}=g*g^n$ et $g^{-n}=\left(g^{-1}\right)^{n}$
\end{itemize}
\end{myrem}

\begin{myexer}
Montrer que pour tout $m, n \in \mathbb{Z}$ on a $g^{m+n}=g^m*g^n$ et $g^{-n}=\left(g^{-1}\right)^n$
\end{myexer}

\begin{myexmpl}
	\leavevmode
	\begin{enumerate}
		\item $G=\mathbb{Z}, *=+$
		\item Soit $E$ un espace vectoriel, $(E, +)$
		\item $(\mathbb{C}^*, \times)$ et $(\mathbb{C}, +)$
		\item Si $\mathbb{K}$ est un corps, $(\mathbb{K}, *)$

		Ces exemples sont des groupes abéliens (c'est à dire commutatifs), les suivants n'en sont pas.
		
		\item Soit $(G, \cdot)$ un groupe fini, on définit $\func{\otimes}{\mathbb{Z}^G \times \mathbb{Z}^G}{\mathbb{Z}^G}{(f_1,f_2)}{\left(g \longmapsto \displaystyle \sum_{h \in G} f_1(h)f_2(h^{-1}*g)\right)}$
		
		\begin{myexer}
			Montrer que $\mathbb{Z}^G$ muni de cette opération n'est pas un groupe mais que $\otimes$ est une loi associative.
		\end{myexer}
		\item $GL_n(\mathbb{R})$ muni de la multiplication de matrices.
	\end{enumerate}
\end{myexmpl}

\begin{myproposition}
	Soit E un ensemble non-vide, on note $\mathfrak{S}(E)$ l'ensemble des applications bijectives de $E$ dans $E$ et $(\mathfrak{S}(E), \circ)$ est un groupe.
\end{myproposition}

\section{Sous-groupe}

\begin{mydef}
	Soit $(G, *)$ un groupe, on appelle \textit{sous-groupe} de $G$ toute partie $H \subseteq G$ munie de $*$ telle que $e \in H$, $\forall (h_1,h_2) \in H^2, h_1*h_2 \in H$  et $\forall h \in H, h^{-1} \in H$.
	On note $H \leqslant G$
\end{mydef}

\begin{myexmpl}
	\begin{enumerate}
		\item Si $(G, *)$ est un groupe alors $\{e\} \leqslant G$
		\item On définit $SL_n(\mathbb{R})=\{M \in \mathcal{M}_n | \det M = 1\}$ le \textit{groupe spécial linéaire} qui est un sous-groupe de $GL_n(\mathbb{R})$
		\item On définit $\mathcal{O}_n(\mathbb{R})=\{M \in \mathcal{M}_n | \,^tMM = I_n\}$ le \textit{groupe orthogonal} qui est un sous-groupe de $GL_n(\mathbb{R})$
		\item $\mathbb{U} = \{z \in \mathbb{C} ~ | ~ |z| = 1\} \leqslant (\mathbb{C}^*, \times)$
		\item Pour $n > 0$, $\mathbb{U}_n = \{z \in \mathbb{C}^* ~ | ~ z^n = 1\} \leqslant \mathbb{U} \leqslant \mathbb{C}^*$
	\end{enumerate}
\end{myexmpl}

\begin{myproposition}
	\leavevmode
	\begin{enumerate}
		\item Soit $n \in \mathbb{Z}$, $n\mathbb{Z} \leqslant \mathbb{Z}$
		\item Tout sous-groupe de $\mathbb{Z}$ est de cette forme
	\end{enumerate}
\end{myproposition}

\begin{myproof}
	\leavevmode
	\begin{enumerate}
		\item $n\mathbb{Z} \subseteq \mathbb{Z}$, $0 \in \mathbb{Z}$, $xn + yn = (x+y)n \in n\mathbb{Z}$ et $-(xn) \in n\mathbb{Z}$
		\item Soit $H \leqslant \mathbb{Z}$, si $H = \{0\}$ \checkmark
		
		Soit $n = min\{h \in H ~|~ h > 0\}$ (il existe par la propriété de la borne supérieure), montrons $H = n\mathbb{Z}$

		\begin{proofpart}{$nZ \subseteq H$}
		\end{proofpart}

		\begin{proofpart}{$nZ \subset H$}\\			
			Soit $h \in H$, on considère sa division euclidienne par n : $h = nq + r$ avec $0 \leqslant r < n$.
			$h - nq = r\in H$, et $n$ est le plus petit élément non-nul, donc $r = 0$.
		\end{proofpart}
		
		\cqfd
	\end{enumerate}
\end{myproof}

\begin{mylemma}
	Soit $G$ un groupe et $\left(H_i\right)_i \in I$ une famille de sous-groupes de $G$, alors $\displaystyle \bigcap_{i \in I}H_i \leqslant G$
\end{mylemma}

\begin{mydef}
	Soit $G$ un groupe et $A$ une partie de G, l'intersection des sous-groupes de $G$ contenant $A$ est appelée \textit{sous-groupe engendré par $A$} et notée $\left\langle A \right\rangle$.
\end{mydef}

\begin{myproperty}
	\leavevmode
	\begin{itemize}
		\item $A \subseteq \langle A \rangle \leqslant G$
		\item Si $H$ est un sous-groupe contenant $A$, alors $\langle A \rangle \subseteq H$
	\end{itemize}
\end{myproperty}

\begin{myexer}
	Montrer que $\langle A \rangle$ est l'unique sous-groupe vérifiant ces propriétés.
\end{myexer}

\begin{myproperty}
	Soit $G$ un groupe et $g \in G$, $\left\langle\{g\}\right\rangle = \langle g \rangle = \{g^n, n \in \mathbb{Z}\}$
\end{myproperty}

\begin{myexer}
	Le démontrer.
\end{myexer}

\begin{myproperty}
	Soit $A$ une partie de $G$, $\langle A \rangle$ est l'ensemble des éléments de la forme $a_1^{n_1}*a_2^{n_2}*...*a_p^{n_p}$ où $a_i \in A$ et $n_i \in \mathbb{Z}$.
\end{myproperty}

\begin{myproof}
	On pose $K$ l'ensemble des éléments de cette forme.
	Montrons
	\begin{enumerate}
		\item $A \subseteq K$ et $K \leqslant G$
		\item Pour tout $H \leqslant G$ tel que $A \subseteq H$ on a $K \subseteq H$
	\end{enumerate}
\end{myproof}

\begin{myexmpl}
	\begin{enumerate}
	\item Soient $k$ et $n$ deux entiers relatifs, $\langle k, n \rangle = k\mathbb{Z} + n\mathbb{Z} = (k \land n) \mathbb{Z}$ d'après l'identité de Bézout.
	\item Soit $n > 0$, si $M \in \mathcal{O}_n(\mathbb{R})$, alors il existe $P \in \mathcal{O}_n(\mathbb{R})$ et $r, s \in \mathbb{N}$, $\theta_1, ...\theta_p \in \mathbb{R}$ tels que $P^{-1}MP$ soit diagonale par blocs :
	
	$$
		\left(
		\begin{array}{ccccc}
			I_r &  &  &  & \\
			 & -I_s & & &\\
			 & & R(\theta_1) & &\\
			 & & & \ddots & \\
			 & & & & R(\theta_p)\\
		\end{array}
		\right)
	$$
	\end{enumerate}
\end{myexmpl}

\begin{myexer}
	Montrer que $\mathcal{O}_n(\mathbb{R})$ est engendré par les réflexions, c'est à dire les matrices orthogonales semblables à $\left(\begin{array}{cccc}
		-1 &&&\\
		&1&&\\
		&&\ddots&\\
		&&&1\\
	\end{array}\right)$ en base orthonormée.
\end{myexer}

\section{Ordre d'un élément dans un groupe}

Soit $g \in G$, on suppose qu'il existe $n > 0$ tel que $g^n=e$.
On a alors $\langle g \rangle = \{e, g, g^2, g^3, ..., g^{n-1}\}$, en effet pour tout $k > 0$, de division euclidienne $k=nq+r$ avec $0 \leqslant r < n$, on a $g^k=g^{nq+r}=e^q g^r=g^r$ d'où $g^r \in \{e, g, g^2, ..., g^{n-1}\}$.

\begin{mydef}
	Soit $g \in G$, on définit \textit{l'ordre} de $g$ par $d=\min \{k > 0 ~ | ~ g^k = e\}$, on a ainsi que $e, g, g^2, ..., g^d$ sont deux à deux distincts. On en conclut que $\langle g \rangle = \{g^k ~ | ~ 0 \leqslant k < d\}$ est de cardinal $d$.
\end{mydef}

En effet $0 \leqslant k \leqslant l < d$, on a $g^l=g^k \Longrightarrow g^{l-k}=e$.

Or $0 \leqslant l - k < d$ et par minimalité de $d$, $l=k$.

\begin{myexmpl}
	\leavevmode
	\begin{enumerate}
		\item Dans $(\mathbb{U}, \times)$, pour $n > 0$ on a $g=\exp\left(\frac{2i \pi}{n}\right)$
		
		g est d'ordre fini égal à $n$.
		
		\item Dans $GL_n(\mathbb{R})$ $\left(\begin{array}{cc}
			1 & 0 \\
			0 & -1 \\
		\end{array}\right)$ est d'ordre 2 et $\left(\begin{array}{cc}
					0 & -1 \\
					1 & 0 \\
				\end{array}\right)$ d'ordre 4.
	\end{enumerate}
\end{myexmpl}

\begin{mythm}
	Soit $G$ un groupe fini et $H \leqslant G$, alors $|H|$ divise  $|G|$. 
\end{mythm}

\begin{mycor}
	Soit $g$ un élément d'un groupe fini, $g$ est d'ordre fini divisant $|G|$.
\end{mycor}

\begin{myexmpl}
	Dans $\mathbb{U}_6$, d'ordre 6, les éléments peuvent avoir pour ordre 1, 2, 3 et 6.
\end{myexmpl}

\begin{myproperty}
	$d > 0$ est l'ordre de $g$ si et seulement si $g^d=e$ et pour tout diviseur strict $k$ de $d$ on a $g^k \neq e$.
\end{myproperty}

\begin{myexer}
	Le démontrer.
\end{myexer}

\begin{myrem}
	Si $g \in G$ et $p$ est un nombre premier tel que $g^p=e$, alors $g=e$ ou l'ordre de $g$ est $p$.
\end{myrem}

\section{Homomorphisme de groupe}

\subsection{Définition}

\begin{mydef}
	Soient $G$ et $G'$ deux groupes, un homomorphisme de $G$ dans $G'$ est une application de $G$ dans $G'$ tel que .
\end{mydef}

\begin{myexmpl}
	\leavevmode
	\begin{enumerate}
		\item On considère $\funcinline{\varphi}{(\mathbb{R}, +)}{(\mathbb{R}^*_+, \times)}{x}{\exp(x)}$
		
		On a $\forall x, y \in \mathbb{R}, ~ \exp(x+y)=\exp(x)\exp(y)$
		
		$\ln$ est l'application réciproque.
		
		\item Le déterminant est un homomorphisme de $(GL_n(\mathbb{R}), \times)$ dans $(\mathbb{R}^*, \times)$
	\end{enumerate}
\end{myexmpl}

\begin{myrem}
	Un homomorphisme $f$ vérifie 
	\begin{itemize}
		\item $f(e)=e'$, car $f(e)=f(e \cdot e)=f(e)f(e)$ puis en simplifiant : $e' = f(e)$
		\item Pour tout $g \in G$, $f(g^{-1})=f(g)^{-1}$
	\end{itemize}
\end{myrem}

\subsection{Étude des homomorphismes de $\mathbb{Z}$}

Soit $(G, \star)$ un groupe et un homomorphisme  $\funcshort{f}{(\mathbb{Z}, +)}{(G, \star)}$, on a :
\begin{itemize}
	\item $f(1) \in G$
	\item $\displaystyle \forall n > 0, ~ f(n) = f\left(\sum_{i=1}^{n} 1\right) = \prod_{i=1}^{n} f(1) = f(1)^n$
	
	et $\displaystyle \forall n > 0, ~ f(-n) = f(n)^{-1} = \left(f(1)^n\right)^{-1} = f(1)^{-n}$
\end{itemize}

On en déduit immédiatement que pour tout homomorphismes $\funcshort{f_1, f_2}{(\mathbb{Z}, +)}{(G, \star)}$

$$f_1=f_2 \Longleftrightarrow f_1(1)=f_2(1)$$

\begin{mythm}
	Soit $(G, \star)$ un groupe, l'application
	$$
	\func{\varphi}{\mathcal{H}(\mathbb{Z}, G)}{G}{f}{f(1)}
	$$
	est bijective et d'application réciproque
	
	$$
	\func{\varphi^{-1}}{G}{\mathcal{H}(\mathbb{Z}, G)}{g}{n \longmapsto g^n}
	$$
\end{mythm}

\subsection{Compositions et isomorphismes}

\begin{mydef}
	\leavevmode
	\begin{itemize}
		\item Un homomorphisme bijectif est appelé \textit{isomorphisme}.
		\item Deux groupes sont dits \textit{isomorphes} si et seulement s'il existe un isomorphisme entre eux.
		\item Un endomorphisme bijectif est un \textit{automorphisme}.
	\end{itemize}
\end{mydef}

\begin{myproperty}
	\leavevmode
	\begin{itemize}
		\item La composée de deux homomorphismes est un homomorphismes.
		\item Si un homomorphisme est bijectif, alors son application réciproque est un homomorphisme.
		\item Soit $G$ un groupe, $Aut(G) \leqslant \mathfrak{S}_G$.
	\end{itemize}
\end{myproperty}

\begin{myexmpl}
	\leavevmode
	\begin{enumerate}
		\item
		$\left\{\begin{array}{rcl}
			\{\pm 1\} & \longrightarrow & Aut(\mathbb{Z}) \\
			1 & \longmapsto & id_{\mathbb{Z}} \\
			-1 & \longmapsto & -id_{\mathbb{Z}}
		\end{array}\right.$
	
	\item Soit $k > 0$, 
		$\left\{\begin{array}{rcl}
			\mathbb{Z} & \longrightarrow & \mathbb{Z} \\
			n & \longmapsto & kn \\
		\end{array}\right.$ est un endomorphisme mais pas un automorphisme.
	\end{enumerate}
\end{myexmpl}

\subsection{Sous-groupes associés à un homomorphisme}

\begin{myproposition}
	Soit $\funcshort{f}{G}{H}$ un homomorphisme de groupes
	\begin{itemize}
		\item Pour tout groupe $H' \leqslant H$, on a $f^{-1}(H') \leqslant G$
		
		\item Pour tout groupe $G' \leqslant G$, on a $f(G') \leqslant H$
	\end{itemize}
\end{myproposition}

\begin{mydef}
	Pour tout homomorphisme $f$ d'un groupe $G$ dans un autre $G'$, on appelle \textit{noyau de $f$} l'ensemble $\ker(f)=\{g \in G ~ | ~ f(g) = e\} \leqslant G$ et l'\textit{image de $f$} l'ensemble $Im(f)=f(G)$
\end{mydef}

\begin{myexer}
	Soient $d, n > 0$, déterminer l'image et le noyau de l'homomorphisme :
	
	$$\left\{
		\begin{array}{ccc}
			\mathbb{U}_n & \longrightarrow & \mathbb{U}_n \\
			x & \longmapsto & x^d
		\end{array}
	\right.$$
\end{myexer}

\begin{mythm}
	Soit $\funcshort{f}{G}{H}$ un homomorphisme de groupes, l'application
	
	$$\varphi ~ : ~ \left\{\begin{array}{rcl}
		\{\text{Sous-groupes de } Im(f)\} & \longrightarrow & \{\text{Sous-groupes de } G \text{ contenant }\ker(f)\} \\
		H & \longmapsto & f^{-1}(H')
	\end{array}\right.$$
	
	est une bijection.
\end{mythm}

\begin{myproof}
	Pour tout $G' \leqslant G$ contenant $\ker f$, on définit $\theta$ par $\theta(G')=f(G') \leqslant Im(f)$.
	
	On démontre que $\theta$ est l'application réciproque de $\varphi$.
	
	\begin{enumerate}
		\item Soit $H' \leqslant Im(f)$, on vérifie $(\theta \circ \varphi)(H')=f(f^{-1}(H'))=H'$ car la co-restriction de $f$ à $Im(f)$ est surjective.
		
		\item Soit $G' \leqslant G$ tel que $\ker f \subseteq G'$, on a $G' \subseteq (\varphi \circ \theta)(G')=f^{-1}(f(G'))$, montrons maintenant $f^{-1}(f(G')) \subseteq G'$ :
		
		Soit $x \in f^{-1}(f(G'))$, alors $f(x) \in f(G')$ et il existe $u \in G'$ tel que $f(x)=f(u)$.
		
		On a donc : $$f(x)=f(u)$$
		$$f(xu^{-1})=e$$
		$$xu^{-1} \in \ker f \subseteq G'$$
		$$x \in \underbrace{\left(G'\right) \cdot u}_{G' \text{ car } u \in G'}=G'$$
		
		Ainsi $f^{-1}(f(G')) \subseteq G'$, et donc $(\varphi \circ \theta)(G')=G'$.
	\end{enumerate}
	
	$\theta$ est bien la bijection réciproque de $\varphi$.
	
	\cqfd
\end{myproof}

\part{Opérations de groupes}

\section{Rappels sur les relations d'équivalence}

\begin{mydef}Soit $X$ un ensemble.

	Une relation binaire $\sim$ sur $X$ est une \textit{relation d'équivalence} si :
	\begin{enumerate}
		\item $\sim$ est transitive : $\forall x, y, z \in X, ~ (x \sim y \land y \sim z \Longrightarrow x \sim z)$
		\item $\sim$ est réflexive : $\forall x \in X, ~ x \sim x$
		\item $\sim$ est symétrique : $\forall x, y \in X, ~ x \sim x$
	\end{enumerate}
	
	On appelle la \textit{classe d'équivalence de $x$} l'ensemble $\{y \in X ~ | ~ x \sim y\}$ et on note $X/\sim$ l'ensemble des classes d'équivalence que l'on appelle \textit{ensemble quotient}.
	
	L'application de $X \longrightarrow X/\sim$ qui à tout élément $x$ associe sa classe d'équivalence est appelée \textit{surjection canonique}.
\end{mydef}

\begin{myproposition}
	Pour une relation $\sim$ donnée, $X/\sim$ est une partition de $X$.
\end{myproposition}

\begin{myexmpl}
	Soit $(G, \cdot)$ un groupe.
	
	\begin{enumerate}
		\item Soit $H \leqslant G$ et $\sim$ la relation définie par $\forall g_1, g_2 \in G, ~ (g_1 \sim g_2 \Longleftrightarrow \exists h \in H ~ : ~ g_1 \cdot h=g_2)$
		
		\item $\mathcal{R}$ par $\forall g_1, g_2 \in G, ~ (g_1 \mathcal{R} g_2 \Longleftrightarrow \exists h \in H ~ g_1 = h g_2 h^{-1})$
		
		\item $\forall H, K \leqslant G, ~ (H \sim K \Longleftrightarrow \exists g \in G ~ : ~ gHg^{-1}=K)$
	\end{enumerate}
\end{myexmpl}

\begin{mythm}
	Soit $\sim$ une relation d'équivalence, sur un ensemble $X$, $Y$ un ensemble et $\funcshort{f}{X}{Y}$ une application constante sur les classes d'équivalences.
	
	Il existe alors une unique application de $\funcshort{g}{(X/\sim)}{Y}$ telle que $g\circ \pi=f$ où $\pi$ est la surjection canonique.
\end{mythm}

\section{Opérations de groupes}

\begin{mydef}
	On définit une action de groupe $(G, \star)$ sur un ensemble $X$ par la donnée d'une application $$\func{\phi}{G \times X}{X}{(g,x)}{g \cdot x}$$
	vérifiant $\forall x \in G, ~ e \cdot x = x$ et $\forall g, h \in G, ~ \forall x \in X, ~ g \cdot (h \cdot x) = (h \star g) \cdot x$
\end{mydef}

\begin{myexmpl}
	Soit $G$ un groupe.
	\begin{enumerate}
		\item Soit $H$ un sous-groupe de $G$, l'action de $H$ sur $G$ définie par $h\cdot g = hg$ est appelée action de $H$ sur $G$ par \textit{translation à gauche}.
		
		\item L'action de $G$ sur lui même définie par $h\cdot g = hgh^{-1}$ est appelée action de $G$ sur lui-même par \textit{conjugaison}.
		
		\item L'action de $G$ sur $\mathcal{S}(G)$ définie par $g \cdot H=gHg^{-1}=i_g(H)$ est l'opération de $G$ sur ses sous-groupes par \textit{conjugaison}.
	\end{enumerate}
\end{myexmpl}

\begin{myproposition}
	Soient $(G, \star)$ un groupe, $X$ un ensemble et $\funcshort{\varphi}{G \times X}{X}$.
	
	$\varphi$ définit une action de groupe si et seulement si pour tout $g \in G$, l'application $\func{\varphi_g}{G}{X}{x}{g \cdot x}$ est bijective et $g\longmapsto \varphi_g$ est un homomorphisme de $(G, \star)$ dans $(\mathfrak{S}_X, \circ)$.
\end{myproposition}

\begin{myrem}
	On en déduit $\forall g, h \in G, ~ \varphi_g \circ \varphi_h = \varphi_{g \star h}$ et $\varphi_{g^{-1}}=(\varphi_g)^{-1}$
\end{myrem}

\begin{myexmpl}
	\begin{enumerate}
		\item L'action de $GL_n(\mathbb{K})$ sur $\mathbb{K}^n$ définie par $M\cdot x=Mx$ est une action de groupe.
	
		\item L'action de $GL_n(\mathbb{K})$ sur $M_{m \times n}(\mathbb{K})$ définie par $M \cdot N=MN$ est une action de groupe.
				
		\item L'action des applications linéaires inversibles de $E$ sur les formes quadratiques de $E$ définie par $g \cdot q = q \circ g^{-1}$
		
		\begin{myexer}
			Le démontrer.
		\end{myexer}
		
		\item L'action par conjugaison des matrice inversibles sur les matrices carrées est une action de groupe.
	\end{enumerate}
\end{myexmpl}

\section{Orbites, stabilisateurs}

\begin{mydef}
	Soit $G$ un groupe opérant sur $X$ et $x \in X$, on définit :
	
	\begin{itemize}
		\item L'orbite de $x$ l'ensemble $G \cdot x=\{g\cdot x ~ | ~ g \in G\}$
		
		\item Le stabilisateur de $x$ l'ensemble $Stab_G(x)=G_x = \{g \in G ~ | ~ g \cdot x = x\}$
	\end{itemize}
\end{mydef}

\begin{myexmpl}
	Considérons l'action de $G$ sur lui-même par translation à gauche.	
	On a $G \cdot x = G$ et $G_x=\{e\}$
\end{myexmpl}

\begin{myproposition}
	La relation sur $X$ définie par :
	$$\forall x, y \in X, x \sim y \Longleftrightarrow \exists g \in G ~ : ~ g \cdot x = y$$
	est une relation d'équivalence, dite \textit{associée à l'opération de $G$ sur $X$}.
\end{myproposition}

\begin{myrem}
	Si $x \in X$, alors $G \cdot x$ est la classe d'équivalence de $x$.
	
	On rappelle aussi que si $\mathcal{R}$ est une relation d'équivalence sur un ensemble $X$, alors $X / \mathcal{R}$ est une partition de $X$.
	
	En particulier, dans le cas d'une relation d'équivalence associée à une action de groupe, $x \sim y \Longleftrightarrow G \cdot x = G \cdot y$.
\end{myrem}

On note alors l'ensemble quotient $X / \sim$ par $G \backslash X$ et on l'appelle ensemble quotient de $X$ par $G$.

\begin{myrem}
	Si on se donne une action à droite sur $X$, on note $X/G$ l'ensemble des classes d'équivalence.
\end{myrem}

\begin{mydef}
	Soit $H \leqslant G$, on note l'ensemble quotient de $G$ par l'opération de translation à gauche de $H$ :
	
	$$H \backslash G = \{Hg ~ | ~ g \in G\}$$
	
	De même on note l'ensemble quotient de $G$ par l'opération de translation à droite de $H$ :
	
	$$G / H = \{gH ~ | ~ g \in G\}$$
\end{mydef}

\begin{mydef}
	Une action de $G$ sur $X$ est dite :
	\begin{itemize}
		\item \textit{transitive} s'il n'existe qu'une seule orbite
		\item \textit{fidèle} si $\forall g \in G, ~ (\forall x \in X, ~ g \cdot x = x) \Longrightarrow g=e$
		\item \textit{libre} si $\forall x \in X, \forall g \in G, g \cdot x = x \Longrightarrow g=e$
	\end{itemize}
\end{mydef}

\begin{myrem}
	L'action est fidèle si : $$\bigcap_{x \in X} Stab(x) = \{e\}$$
	et elle est transitive si $\forall x \in X, ~ Stab(x) = \{e\}$.
\end{myrem}

\begin{myexer}
	Une opération est fidèle si et seulement si l'homomorphisme associé est injectif.
\end{myexer}

\begin{myexmpl}
	\leavevmode
	\begin{enumerate}
		\item L'action de $GL_n(\mathbb{R})$ sur  $\mathbb{R}^n$ admet deux orbites : $\{0\}$ et $\mathbb{R}^n \backslash \{0\}$
		
		\item L'action de $GL_n(\mathbb{R})$ sur  $\mathbb{R}^n \backslash \{0\}$ admet une unique orbite $\mathbb{R}^n \backslash \{0\}$ donc l'action est transitive.
		
		Elle est fidèle ($(\forall x \in \mathbb{R}^n, ~ Mx=x) \Longrightarrow M=I_n$) mais elle n'est pas libre car une matrice inversible différente de $I_n$ de vecteur propre $x$ et de valeur propre $1$ vérifie $Mx=x$.
		
		\item Soient $H \leqslant G$, l'action de $H$ sur $G$ par translation à gauche.
		
		L'action est transitive si et seulement si $G = H$.
		
		L'opération est libre car pour tout $g \in G$ et $x \in G$, si on a $g \star x=x$, alors en simplifiant par $x$ on a $g=e$. Elle est donc également fidèle.
		
		\item On considère les sommets du cube de côté $2$ l'ensemble $\mathcal{C}=\{(x_1, x_2, x_3) \in \mathbb{R}^3 ~ | ~ \forall i, ~ |x_i| = 1 \}$ et le groupe $G \leqslant SO_3(\mathbb{R}^3)$ préservant globalement $\mathcal{C}$.
		
		L'opération est fidèle, car si $g$ est une application fixant trois points distincts, alors elle possède trois vecteurs propres linéairement indépendants de valeur propre 1, elle est donc diagonalisable et est égale à l'identité.
		
		Elle est transitive car tout sommet peut être envoyé sur un autre.
		
		Elle n'est pas libre car la rotation autour d'une "diagonale" fixe les sommets par laquelle elle passe.
	\end{enumerate}
\end{myexmpl}

\subsection{Aspects numériques}

\begin{myproposition}
	\leavevmode
	\begin{enumerate}
		\item Étant donné une opération de $G$ sur $X$, il existe une application bijective
		
		$$\func{\varphi}{G/Stab(x)}{G \cdot x}{g Stab(x)}{g \cdot x}$$
		
		\item Si $G$ et $X$ sont finis, alors $$|G \cdot x| = \frac{|G|}{|Stab(x)|}$$
		
		\item Si $G$ et $X$ sont finis, on choisit pour chaque orbite $\omega$ un élément $x_\omega \in \omega$ et on obtient :
		
		$$|X| = \sum_{\omega \in G \backslash X} \frac{|G|}{Stab_G(x_\omega)}$$
	\end{enumerate}
	
	Les deux égalités s'appellent \textit{équations aux classes}.
\end{myproposition}

\begin{myproof}
	Soit $x \in G$, on considère l'action de $Stab_G(x)$ sur $G$ par translation à droite, et $\sim$ la relation d'équivalence sur $G$ associée à celle-ci.
	
	On définit : $$\func{f}{G}{G \cdot x}{g}{g \cdot x}$$
	
	qui est constante sur les classes d'équivalences de $G/\sim$ (de la forme $g\text{Stab}_G(x)$), donc il existe $\funcshort{\varphi}{G/Stab_G(x)}{G \cdot x}$ telle que $\forall g \in G, ~ \varphi(gStab_G(x))=g \cdot x$
	
	Montrons que $\varphi$ est bijective.
	
	\begin{proofpart}{$\varphi$ est surjective}
	
		Soit $y \in G \cdot x$, il existe $g \in G$ tel que $y = g \cdot x = f(g) = \varphi(g Stab_G(x))$.
	\end{proofpart}
	
	\begin{proofpart}{$\varphi$ est injective}
	
		Soient $\alpha, \beta \in G/Stab_G(x)$ tels que $\varphi(\alpha) = \varphi(\beta)$.
		
		Il existe $g, h \in G$ tels que $\alpha = g Stab_G(x)$ et $\beta = h Stab_G(x)$.
		
		Alors $g \cdot x=f(g)=\varphi(\alpha)=\varphi(\beta)=f(h)=h \cdot x$, d'où $h^{-1} \star g \in Stab_G(x)$ ainsi $gStab_G(x)=hStab_G(x)$, on obtient alors $\alpha = \beta$.
	\end{proofpart}
	
	\cqfd
\end{myproof}

\begin{myexer}
	Pour chaque classe $\omega \in G/Stab_G(x)$ on pose $g_\omega \in \omega$, alors pour tout $g \in G$ il existe un unique couple $(\omega, h) \in G/Stab_G(x) \times Stab_G(x)$ tel que $g = g_\omega \star h$
\end{myexer}

\begin{mythm}
	Soit $(G, \star)$ un $p$-groupe (avec $p$ premier) opérant sur un ensemble fini $X$, on note
	$X^G = \{x \in X ~ | ~ \forall g \in G, ~ g \cdot x = x\}$, on a :
	
	$$|X^G| \equiv |X| \mod p$$
\end{mythm}

\begin{myproof}
	On remarque tout d'abord
	$$\forall x \in X, (x \in X^G \Longleftrightarrow G \cdot x = \{x\})$$
	
	et de manière équivalente sachant qu'une orbite n'est jamais vide
	
	$$\forall x \in X, (x \notin X^G \Longleftrightarrow |G \cdot x| \geqslant 2)$$
	
	Soient $\omega_1, \omega_2, ...\omega_n$ les orbites de $X$ sous l'actions de $G$ ordonnées de telle façon que les $k$ premières soient les orbites réduites à un élément, où $k = \left|X^G\right|$, et $x_1, x_2, ...x_n$ des éléments de $X$ tels que $\forall i \leqslant n, ~ x_i \in \omega_i$.
	
	Les orbites de $(\omega_i)_{i \leqslant n}$ forment une partition de $X$, d'où :
	
	$$|X| = \sum_{i = 1}^{n} |\omega_i|$$
	
	$$|X| = \sum_{i = 1}^{k} |w_i| + \sum_{i = k + 1}^{n} |w_i|$$
	
	$$|X| = \sum_{i = 1}^{k} 1 + \sum_{i = k + 1}^{n} |w_i|$$
	
	$$|X| = \left|X^G\right| + \sum_{i = k + 1}^{n} |w_i|$$
	
	Chaque $\text{Stab}(x_i)$ est un sous-groupe de $G$, qui est d'ordre puissance de $p$, donc d'après le théorème de Lagrange, $\text{Stab}(x_i)$ l'est aussi, et ainsi $\displaystyle |\omega_i| = \frac{|G|}{|\text{Stab}(x_i)|}$ est une puissance de $p$.
	
	De plus, pour tout $i > k$, $x_i \notin X^G$ donc $|G \cdot x_i| = |w_i| \geqslant 2$, et comme $p \geqslant 2$, $|\omega_i|$ est une puissance non-nulle de $p$.
	
	On en déduit que $\displaystyle \sum_{i = k + 1}^{n} |w_i|$ est un multiple de $p$, d'où :
	
	$$|X| \equiv \left|X^G\right| \mod p$$
	
	\cqfd
\end{myproof}

\begin{mythm}Théorème de Cauchy

	Soit $G$ un groupe fini et $p$ un diviseur premier de $|G|$, alors il existe $g \in G$ tel que $|g|=p$.
\end{mythm}

\begin{myproof}
	Soit $n = |G|$ et soit $X = \{(g_1, g_2, ...g_p) ~ | ~ g_1 g_2 ... g_p = e\}$, cet ensemble est de cardinal $n^{p-1}$ car il existe $n^{p-1}$ $(p-1)$-uplets $(g_1, ..., g_{p-1})$ de $G$ et il existe pour chacun un unique $g_p = (g_1...g_{p-1})^{-1}$ tel que $g_1...g_p = e$.
	
	On pose $\omega$ la permutation de $X$
	
	$$\func{\omega}{X}{X}{(g_1, g_2, ... g_p)}{(g_2, g_3, ... g_p, g_1)}$$
	
	Celle-ci est bien définie car pour tout $(g_1, ... g_p) \in X$
	$$g_1...g_p = e$$
	$$g_2...g_p g_1 = g_1^{-1} g_1$$
	$$g_2...g_p g_1 = e$$
	$$\underbrace{(g_2, ..., g_p, g_1)}_{\omega(g_1, ...g_p)} \in X$$
	
	et par récurrence, pour tout $k \geqslant 0$, $\omega^k (g_1, ...g_p) \in X$.
	
	De plus, elle est d'ordre $p$, il  existe donc un unique homomorphisme de $G$ vers $\mathcal{S}_X$ envoyant $e^{2 i \pi / p}$ sur $\omega$, on définit une action de $\mathbb{U}_p$ sur $X$ à l'aide de celle-ci.
	
	D'après le théorème précédent, on a :
	$$\left|X^G\right| \equiv |X| \mod p$$
	$$\left|X^G\right| \equiv n^{p-1} \mod p$$
	$$\left|X^G\right| \equiv 0 \mod p, \text{ car $p$ divise $n$}$$
	
	Or $X^G$ est non-vide, il contient $(e, e, ...e)$, donc $X^G$ est un multiple de $p \geqslant 2$, c'est-à-dire qu'il contient un élément $g=(g_1, g_2, ... g_p)$ tel que $e^{2 i \pi / p} \cdot g = g$, donc $g_1 = g_2$, $g_2 = g_3$, etc. et alors $g_1 = g_2 = ... = g_p$.
	
	$g=(g_1, ..., g_1) \in X^G$ donc $g_1^p = e$ avec $g_1 \neq e$, il existe donc un élément d'ordre $p$ dans $G$.
	
	\cqfd
\end{myproof}

\section{Applications à la géométrie affine}

Soit $\mathbb{K}$ un corps.

\subsection{Espaces affines}

\begin{mydef}
	Soit $\mathcal{E}$ un ensemble non-vide, et $E$ un espace vectoriel, $\mathcal{E}$ est un \textit{espace affine} de \textit{direction} $E$ s'il existe une action libre et transitive de $E$ sur $\mathcal{E}$
	
	$$\anonfunc{E}{\mathcal{E}}{(M, u)}{M + u}$$

	On a alors :

	\begin{itemize}
		\item Pour tout $A, B \in \mathcal{E}$ il existe un unique $u \in E$ tel que $A + u = B$, on note $\overrightarrow{AB} = u$.
		
		\item Pour tout $A, B, C \in \mathcal{E}$ on a :
		\begin{enumerate}
			\item $\overrightarrow{AB} = 0 \Leftrightarrow A = B$
			\item $\overrightarrow{AB}+\overrightarrow{BC} = \overrightarrow{AC}$
			\item $-\overrightarrow{AB} = \overrightarrow{BA}$
		\end{enumerate} 
	\end{itemize}
\end{mydef}

En effet, pour tout $A, B \in \mathcal{E}$, l'action étant transitive, il existe $u \in E$ tel que $B + u = A$, et étant libre, s'il existe un autre $v \in E$ tel que $B = A + u = A + v$ on a que $u = v$.

De plus, pour tout $A, B, C \in \mathcal{E}$ on vérifie les propriétés :
\begin{enumerate}
	\item $\overrightarrow{AB} = 0 \Longleftrightarrow B = A + \overrightarrow{AB} = A \Longleftrightarrow A = B$
	\item $A + (\overrightarrow{AB} + \overrightarrow{BC}) = (A + \overrightarrow{AB}) + \overrightarrow{BC} = B + \overrightarrow{BC} = C$ d'où $\overrightarrow{AB} + \overrightarrow{BC} = \overrightarrow{AC}$
	\item $A = A + 0 = A + \overrightarrow{AA}$ donc $\overrightarrow{AA} = 0$, alors $\overrightarrow{AA} = \overrightarrow{AB} + \overrightarrow{BA} = 0$, donc $-\overrightarrow{AB} = \overrightarrow{BA}$.
\end{enumerate}

\subsection{Sous-espaces affine}

Soit $\mathcal{E}$ un espace affine de direction $E$.

\begin{mydef}
	On appelle \textit{sous-espace affine} de $\mathcal{E}$ tout sous-ensemble $\mathcal{F} \subseteq \mathcal{E}$ tel que $\mathcal{F}$ est l'orbite d'un élément de $\mathcal{E}$ pour l'opération d'un sous-espace vectoriel de $E$.
\end{mydef}

\begin{myexmpl}
	\leavevmode
	\begin{enumerate}
		\item
		$E=\mathcal{E} = \mathcal{K}^3$
		
		$\mathcal{F} = \{ (x, y, z) \in \mathbb{K}^3 ~ | ~ z = 1 \}$
		
		$\mathcal{F} = (0, 0, 1) + \{(x, y, z) \in \mathbb{K}^3 ~ | ~ z = 0\}$
		
		\item
		Soit $V$ un $\mathbb{K}$-espace vectoriel et $\varphi$ une forme linéaire non-nulle, alors $\{u \in V ~ | ~ \varphi(u) = 1\} = u_0 + \ker \varphi$ est un sous-espace affine, où $\varphi(u_0)=1$.
		
		\begin{myexer}
			Le démontrer
			\label{exo11}
		\end{myexer}
	\end{enumerate}
\end{myexmpl}

\begin{myrem}
	Soient $X, Y \in \mathcal{E}$ et $F, G$ des sous-espace vectoriels de $E$ tels que $X + F = Y + G$, alors $F = G$.
	
	$X \in X + F = Y + G$, alors $\overrightarrow{XY} \in G$, symétriquement on montre $\overrightarrow{YX} \in F$ et donc $\overrightarrow{XY} = -\overrightarrow{YX} \in F$, on en déduit $\overrightarrow{XY} \in F \cap G$.
	
	Soit $u \in F$, on a $X + u \in X + F = Y + G$.

	Il existe $v \in G$ tel que
	$$X + u = X + v$$

	$$X + (u-v) = X$$

	$$\overrightarrow{XY} = u - v$$

	$$u = \overrightarrow{XY} + v \in G$$
	
	Donc $F \subseteq G$, symétriquement $G \subseteq F$ donc $F = G$.
\end{myrem}

\begin{myrem}
	Soit $X \in \mathcal{E}$ soit $F$ un sous-espace vectoriel de $E$, alors pour tout $M \in \mathcal{E}$, $M \in X + F \Longleftrightarrow \overrightarrow{XM} \in F$
	
	\begin{myexer}
		Remontrer la remarque précédente à l'aide de cette propriété.
		\label{exo12}
	\end{myexer}
\end{myrem}

\begin{mydef}
	Soit $\mathcal{F}$ un sous-espace affine de $\mathcal{E}$, il existe un unique sous-espace vectoriel $F$ de $E$ tel qu'il existe $M \in \mathcal{E}$ vérifiant $\mathcal{F} = M + F$, on l'appelle direction de $\mathcal{F}$.
	
	Si $\dim F$ est finie, on dit que $\mathcal{F}$ est de dimension $\dim F$.
\end{mydef}

\begin{myproperty}
	\leavevmode

	\begin{itemize}
	\item Soit $I$ un ensemble et $\left(\mathcal{F}_i\right)_{i \in I}$ une famille de sous-espace affine de $\mathcal{E}$ de directions respectives $\left(F_i\right)$.
	
	Si $\displaystyle \bigcap_{i \in I} \mathcal{F}_i \neq \emptyset$, alors il est un sous-espace affine de direction $\displaystyle \bigcap_{i \in I} F_i$.
	
	\item Soient $\mathcal{F}, \mathcal{G}$ deux sous-espaces affines de $\mathcal{E}$ de directions supplémentaires dans $E$, alors $\mathcal{F} \cap \mathcal{G}$ est réduite à un point.
	
	\item Soient $\mathcal{F}$ et $\mathcal{G}$ de directions respectives $F$ et $G$, si $\mathcal{F} \subseteq \mathcal{G}$ alors $F \subseteq G$.
	\end{itemize}
\end{myproperty}

\begin{myproof}
	\leavevmode
	\begin{enumerate}
		\item On se restreint au cas ou $I$ est de cardinal 2, soient $\mathcal{F}$ et $\mathcal{G}$ deux sous-espaces affines de directions $F$ et $G$.
		
		\item Soient $\mathcal{A}, \mathcal{B}$ deux sous-espaces affines de directions supplémentaires.
		
		Il existe $A, B \in \mathcal{E}$ tels que $\mathcal{F} = A + F$  et $\mathcal{G} = B + G$.
		
		$F \oplus G = E$, on peut alors décomposer $\overrightarrow{AB} = f + g$ avec $(f,g) \in F \times G$
		
		$$A + \overrightarrow{AB} = B$$
		
		$$\underbrace{A + f}_{\in \mathcal{F}} = A + (\overrightarrow{AB} - g) = \underbrace{B - g}_{\in \mathcal{G}}$$
		
		Donc $X := A + f \in \mathcal{F} \cap \mathcal{G}$.
		
		Ce point est de plus unique car si $Y \in \mathcal{F} \cap \mathcal{G}$, alors $\overrightarrow{XY} \in F \cap G = \{0\}$, c'est-à-dire $\overrightarrow{XY} = 0$ et donc $X = Y$.
	\end{enumerate}
\end{myproof}

\begin{myexmpl}
	\item Soit $S$ un système d'équations linéaires à $n$ inconnues, soit $S_h$ son système homogène.
	
	Soit $\mathcal{F} \subseteq \mathbb{K}^n$ l'ensemble des solutions de $S$, et $F \subseteq \mathbb{K}^n$ l'ensemble de solutions de $S_h$.
	
	Si $F \neq \emptyset$, alors $\mathcal{F}$ est un sous-espace affine de direction $F$.
\end{myexmpl}

\begin{myrem}
	On suppose $\dim E = 2$, deux droites affines de $\mathcal{E}$ vérifient une et une seule des trois conditions suivantes :
	\begin{itemize}
		\item Elles sont égales
		\item Elles sont disjointes de même direction
		\item Leur intersection est réduite à un point
	\end{itemize}
	
	\begin{myexer}
		Le démontrer
		\label{exo13}
	\end{myexer}
\end{myrem}

\begin{mydef}
	Deux sous-espaces affines sont \textit{parallèles} s'ils sont de même direction.
\end{mydef}

\begin{myrem}
	Deux sous-espace affines parallèles sont soit égaux, soit disjoints.
	
	En effet s'ils ne sont pas disjoints, alors il existe $M \in \mathcal{F} \cap \mathcal{G}$ et donc $\mathcal{F} = M + \overrightarrow{\mathcal{F}} = \mathcal{F} = M + \overrightarrow{\mathcal{G}} = \mathcal{G}$
\end{myrem}

\subsection{Applications affines}

\begin{mydef}
	Soient $\mathcal{E}, \mathcal{F}$ deux espaces affines de directions $E$ et $F$ et $\funcshort{f}{\mathcal{E}}{\mathcal{F}}$, $f$ est \textit{une application affine} s'il existe application linéaire $l$ de $E$ à $F$ telle que :
	
	$$\forall A, B \in \mathcal{E}, \overrightarrow{f(A) f(B)} = l(\overrightarrow{AB})$$
	
	ou de manière équivalente
	
	$$\forall (M, u) \in \mathcal{E} \times E, ~ f(M + u) = f(M) + l(u)$$
\end{mydef}

\begin{myproof}
	$$\forall A, B \in \mathcal{E}, ~ \overrightarrow{f(A)f(B)} = l(\overrightarrow{AB}) \Longleftrightarrow \forall A, B \in \mathcal{E}, ~ f(A) + l(\overrightarrow{AB}) = f(B)$$

	$$\forall A, B \in \mathcal{E}, ~ \overrightarrow{f(A)f(B)} = l(\overrightarrow{AB}) \Longleftrightarrow \forall (A, u) \in \mathcal{E} \times E, ~ f(A) + l(u) = f(A+u)$$
\end{myproof}

\begin{myexmpl}
	\leavevmode
	
	\begin{enumerate}
		\item Soit $u \in E$, on pose $\func{\tau_u}{\mathcal{E}}{\mathcal{E}}{x}{x + u}$, c'est une application affine d'application linéaire associée $id_E$.
		
		\item Soit $\Omega \in \mathcal{E}$ et $\lambda \in \mathbb{K}^{\times}$, on note $h_{\Omega, \lambda}$ de $\mathcal{E}$ dans $\mathcal{E}$ définie par $h_{\Omega \lambda}(M) = \Omega + \lambda \overrightarrow{\Omega M}$, c'est une application affine d'application linéaire associée $\lambda id_E$.
		
		Pour tout $(M, u) \in \mathcal{E} \times E$ :
		
		$\begin{aligned}
		   h(M+u)& = \Omega + \lambda \overrightarrow{\Omega(M+u)} \\
		   & = \Omega + \left(\overrightarrow{\Omega M} + \overrightarrow{M(M+u)}\right) \\
		   & = \Omega + \lambda \overrightarrow{\Omega M} + \lambda u \\
		   & = h(M) + \lambda u \\
		   & = h(M) + (\lambda id_E)(u)
		 \end{aligned}$
		
	\end{enumerate}
\end{myexmpl}

\begin{myrem}
	Soient $X, Y \in \mathcal{E}$ et $u, v \in \mathcal{E}$, on a $\overrightarrow{(X+u)(Y+v)} = \overrightarrow{XY} + v - u$ car
	$$\begin{aligned}
		\overrightarrow{(X+u)(Y+v)}& = \overrightarrow{(X+u)X} + \overrightarrow{XY} + \overrightarrow{Y(Y+v)}	\\
		& = -u + \overrightarrow{XY} + v
	\end{aligned}$$
\end{myrem}

\begin{myproposition}
	Soient $\mathcal{F}$ et $\mathcal{G}$ des espaces affines, alors :
	\begin{enumerate}
		\item $\anonfunc{\mathcal{E}}{\mathcal{E}}{X}{X}$ est une application affine d'application linéaire associée $id_E$.
		
		\item Si $\funcshort{f}{\mathcal{E}}{\mathcal{F}}$ et $\funcshort{f}{\mathcal{F}}{\mathcal{G}}$ sont des applications affines, alors $g \circ f$ l'est aussi, et son application linéaire associée est la composée de celle de $f$ avec celle de $g$.
		
		\item Soit $\funcshort{f}{\mathcal{E}}{\mathcal{F}}$ une application affine, on note $L$ son application linéaire associée, alors $f$ est bijective si et seulement si $L$ bijective, dans ce cas $f^{-1}$ est affine, d'application linéaire associée $L^{-1}$.
	\end{enumerate}
\end{myproposition}

\begin{myproof}
	\leavevmode
	
	\begin{enumerate}
		\item \checkmark
		\item Soient $A, B \in \mathcal{E}$ $$\overrightarrow{g(f(A)) g(f(B))} = L_g(\overrightarrow{f(A) f(B)}) = L_g(L_f(\overrightarrow{AB})) = (L_g \circ L_f)(\overrightarrow{AB})$$
		\item
		\begin{proofpart}{$f \text{ bijective} \Longrightarrow L \text{ bijective}$}
		
			Soit $A \in \mathcal{E}$, et $u \in E$ tel que $L(u)=0$, on a $f(A + u) = f(A) + L(u) = f(A)$
			
			or $f$ est injective, alors $A+u = A$, donc $u = 0$, ainsi $\ker L = \{0\}$, $L$ est donc injective.
			
			Soit $v \in F$ et $u \in E$ tel que $A + u = f^{-1}(f(A) + v)$ alors $$L(u) = \overrightarrow{f(A)f(A+u)} =  \overrightarrow{f(A)(f(A)+v)} = v$$ donc $L$ est surjective.
			
			$L$ est donc bijective.

		\end{proofpart}

		\begin{proofpart}{$L \text{ bijective} \Longrightarrow f \text{ bijective}$}
			
			Réciproquement, si $L$ est bijective, soit $M,X \in \mathcal{E}$ fixés et un point quelconque $A \in \mathcal{E}$, on a
			
			$$f(M) = X \Longleftrightarrow f(A + \overrightarrow{AM}) = f(A) + L(\overrightarrow{AM}) = X$$
			
			$$f(M) = X \Longleftrightarrow L(\overrightarrow{AM}) = \overrightarrow{f(A)X}$$
			
			$$\overrightarrow{AM} = L^{-1}(\overrightarrow{f(A)X})$$
			
			On pose $\func{f'}{\mathcal{F}}{\mathcal{E}}{X}{A + L^{-1}(\overrightarrow{f(A)X})}$
	
			On a bien $f'\circ f = id_{\mathcal{E}}$ et $f \circ f' = id_{\mathcal{F}}$ :
			
			Soit $X \in \mathcal{E}$, on a :
			$$
			\begin{aligned}
				(f' \circ f)(X) &= f'\left(f(A)+L\left(\overrightarrow{AX}\right)\right)\\
				&=A+L^{-1}\left(\overrightarrow{f(A)\left(f(A)+L\left(\overrightarrow{AX}\right)\right)}\right)\\
				&=A+L^{-1}\left(L\left(\overrightarrow{AX}\right)\right)\\
				&=A+\overrightarrow{AX}\\
				&=X
			\end{aligned}$$
			
			De même pour $(f \circ f')(Y)$.
		\end{proofpart}

		On suppose à présent que $f$ et $l$ sont bijectives.

		\begin{proofpart}{$f^{-1} \text{ est affine, d'application linéaire associée } L^{-1}$}
		
		Soient $X, Y \in \mathcal{F}$, on montre que $\overrightarrow{f^{-1}(X)f^{-1}(Y)} = L^{-1}(\overrightarrow{XY})$
		
		$L(\overrightarrow{f^{-1}(X)f{-1}(Y)} = \overrightarrow{f(f^{-1}(X) f(f^{-1}(Y))} = \overrightarrow{XY}$
		
		Or $L$ est injective, donc $\overrightarrow{f^{-1}(X)f^{-1}(Y)} = L^{-1}(\overrightarrow{XY})$
		
		Comme $\funcshort{L}{E}{F}$ est linéaire injective, on a que $L^{-1}$ est linéaire.

		\end{proofpart}		
	\end{enumerate}
\end{myproof}

\begin{mycor}
	\leavevmode
	\begin{itemize}
		\item L'ensemble des applications affines bijectives de $\mathcal{E}$ est un sous-groupe de $\mathfrak{S}_{\mathcal{E}}$, il est appelé \textit{groupe affine de $\mathcal{E}$} et est noté $GA(\mathcal{E})$.
		
		\item L'ensemble des translations est un sous-groupe de $GA(\mathcal{E})$.
		
		\item L'ensemble $G$ des translations et des homothéties de rapport non-nul est un sous-groupe de $GA(\mathcal{E})$.
	\end{itemize}
\end{mycor}

\begin{myexer}
	Soit $f$ d'application linéaire associée $L = \phi(f)$, $f$ est une translation ou une homothétie si et seulement s'il existe $\lambda \in \mathbb{K}^{\times} ~ : ~ L = \lambda id_E$.
	
	Montrons que s'il existe $\lambda \neq 0$ tel que $L = \lambda id_E$, alors $f$ est une homothétie ou une translation :
	
	\begin{proofpart}{Si $\lambda = 1$}

		Soient $M, N \in \mathcal{E}$ :
		$$
		\begin{aligned}
			\overrightarrow{M f(M)} &= \overrightarrow{M f(N)} + \overrightarrow{f(N) f(M)} \\
			& = \overrightarrow{M f(N)} + L\left(\overrightarrow{NM}\right)\\
			& = \overrightarrow{M f(N)} + \overrightarrow{NM} \\
			& = \overrightarrow{N f(N)}
		\end{aligned}
		$$
		
		il existe donc un certain $u \in \mathcal{E}$ tel que $\forall M \in \mathcal{E}, ~ f(M) = M+u$.
		
	\end{proofpart}
	
	
		\begin{proofpart}{Si $\lambda \neq 1$}
	
			Soit $M \in \mathcal{E}$, on pose $\Omega \in \mathcal{E}$ tel que $\displaystyle \overrightarrow{M\Omega} = \frac{\overrightarrow{Mf(M)}}{1 - \lambda}$

			$\Omega$ est un point fixe de $f$ :

			$$
			\begin{aligned}
				f(\Omega) &=f(M + \overrightarrow{M\Omega}) \\
				&=f(M) + \lambda \overrightarrow{M\Omega} \\
				&=f(M) + \frac{\lambda}{1-\lambda} \overrightarrow{Mf(M)}\\
				&= \Omega + \overrightarrow{\Omega M} + \overrightarrow{M f(M)} + \frac{\lambda}{1-\lambda}\overrightarrow{Mf(M)}\\
				&=\Omega - \frac{\overrightarrow{M f(M)}}{1-\lambda} + \overrightarrow{M f(M)} + \frac{\lambda}{1 - \lambda} \overrightarrow{M f(M)} \\
				&=\Omega + \frac{-1 + (1 - \lambda) + \lambda}{1 - \lambda} \overrightarrow{M f(M)}\\
				&=\Omega + 0 \cdot \overrightarrow{M f(M)}\\
				&=\Omega
			\end{aligned}
			$$
			
			On a ainsi pour tout $M \in \mathcal{E}$ :
			
			$$f(M) = f(\Omega + \overrightarrow{\Omega M}) = f(\Omega) + \lambda \overrightarrow{\Omega M} = \Omega + \lambda \overrightarrow{\Omega M}$$
			
		\end{proofpart}
	
\end{myexer}

$G$ est alors l'image réciproque du sous-groupe $\{\lambda id_E ~ | ~ \lambda \neq 0\}$ par $\phi$.

\part{Groupes symétriques}

\begin{myexer}
	Soient $E$ et $F$ deux ensembles, $f$ une bijection de $E$ dans $F$, l'application
	$$\anonfunc{\mathcal{S}_E}{\mathcal{S}_F}{\sigma}{f^{-1} \circ \sigma \circ f}$$
	est un isomorphisme de groupes et elle est bien définie.
\end{myexer}

\begin{mydef}
	Soit $n \geqslant 1$, on appelle $n$-ième groupe symétrique, noté $\mathcal{S}_n$ le groupe $\mathcal{S}_{\{1, 2, ...n\}}$.
\end{mydef}

\begin{myproposition}
	Soit $n \geqslant 1$, on a $\left|\mathcal{S}_n\right| = n!$
\end{myproposition}

\begin{mydef}
	\leavevmode
	\begin{itemize}
		\item Soient $i, j \leqslant n$ distincts, on note $(i ~ j)$ la \textit{transposition} définie par
		$$\left\{
			\begin{array}{l c l}
				i & \longmapsto & j \\
				j & \longmapsto & i \\
				k & \longmapsto & k, ~ \text{ où } k \neq i, j
			\end{array}
		\right.$$
		
		Celle-ci est d'ordre 2.
		
		\item Soient $a_1, ...a_l$ distincts avec $2 \leqslant l \leqslant n$, on note $(a_1 ~ a_2 ... a_l)$ le \textit{$l$-cycle} défini par
		$$\left\{
			\begin{array}{l c l}
				a_1 & \longmapsto & a_2 \\
				a_2 & \longmapsto & a_3 \\
				&...& \\
				a_i & \longmapsto & a_{i+1}, ~ \text{ où } i \leqslant k-1 \\
				a_k & \longmapsto & a_1, ~ \text{ où } k \neq i, j
			\end{array}
		\right.$$
	\end{itemize}
\end{mydef}

\begin{mydef}
	On définit le \textit{support} de $\sigma$ par
	$$supp(\sigma) = \{i \leqslant n ~ | ~ \sigma(i) \neq i\}$$
\end{mydef}

\section{Propriétés de calcul élémentaires}

\begin{myproposition}
	Soient $\sigma, \omega \in \mathscr{S}_n$, si $supp(\sigma) \cap supp(\omega) = \emptyset$ alors $\sigma \omega = \omega \sigma$
\end{myproposition}

\begin{myproof}
	Soit $x \in \{1, ... n\}$
	\begin{itemize}
		\item Si $x \notin supp(\sigma)$ et $x \notin supp(\omega)$, alors $\sigma(\omega(x))=\sigma(x)=x$ et $\omega(\sigma(x))=\omega(x)=x$.
		\item Si $x \in supp(\omega)$, alors 
			\begin{itemize}
				\item $\omega(x) \in supp(\omega)$ et donc $\omega(x) \notin supp(\sigma)$, ainsi $\sigma(\omega(x))=\sigma(x)$
				\item $\omega(\sigma(x))=\sigma(x)$ car $\sigma(x) \notin supp(\omega)$
			\end{itemize}
		\item Le cas où $x \in supp(\sigma)$ est symétrique
	\end{itemize}
	
	On a utilisé le fait que $\sigma(supp(\sigma)) = supp(\sigma)$, on le vérifie en rapidement : si $x \in supp(\sigma)$ alors $\sigma(x) \neq x$ et donc $\sigma(\sigma(x)) \neq \sigma(x)$ par injectivité, d'où $\sigma(x) \in supp(\sigma)$, c'est-à-dire $\sigma(supp(\sigma)) \subseteq supp(\sigma)$ et comme $\sigma$ est bijective, $\sigma(supp(\sigma)) = supp(\sigma)$.
	
	\cqfd
\end{myproof}

\begin{myproposition}
	Soient $x_1, x_2, ... x_l$ des éléments de $\{1, ..n\}$ deux à deux distincts, alors :
	
	$$(x_1 ~ x_2)(x_2 ~ x_3) ... (x_{l-1} ~ x_l) = (x_1 ~ x_2~ ...~ x_l)$$
	
	et pour tous $\sigma \in \mathscr{S}_n$ : $\sigma (x_1 ~x_2 ~...~ x_l) \sigma^{-1} = (\sigma(x_1) ~ \sigma(x_2)~ ... ~\sigma(x_l))$
\end{myproposition}

\begin{myproof}
	Pour toute suite d'éléments distincts $a_1, a_2, ... a_k$ et tout $i \leqslant k$, on pose $c = (a_1 ~ a_2 ~ ... ~ a_i)$ et $c' = (a_i ~ a_{i+1} ~ ... ~ a_k)$ et on a :
	\begin{itemize}
		\item $(c \circ c')(a_1) = c(a_1)=a_1$
		\item $(c \circ c')(a_2) = c(a_2) = a_2$
		
		...
		
		\item $(c \circ c')(a_i) = c(a_{i+1})=a_{i+1}$
		\item $(c \circ c')(a_{i+1}) = c(a_{i+2})=a_{i+2}$
		
		...
		
		\item $(c \circ c')(a_{k}) = c(a_{i})=a_1$
	\end{itemize}

	Donc $(a_1 ~ a_2 ~ ... ~ a_i)(a_i ~ a_{i+1} ~ ... ~ a_{n})$
	
	On utilise ce résultat pour montrer par récurrence la première égalité :
	
	$$
	\begin{aligned}
		(x_1 ~ x_2)(x_2 ~ x_3) ... (x_{l-1} ~ x_l) &= (x_1 ~ x_2 ~ x_3)(x_3 ~ x_4) ... (x_{l-1} x_l) \\
		&= (x_1 ~ x_2 ~ x_3 ~ x_4)...(x_{l-1} ~ x_l) \\
		... \\
		&= (x_1 ~ x_2 ~ ... x_i)(x_i ~ x_{i+1})(x_{i+1} ~ x_{i+2})...(x_{l-1} ~ x_l) \\
		&= (x_1 ~ x_2 ~ ... ~ x_{i+1})(x_{i+1} ~ x_{i+2})...(x_{l-1} ~ x_l) \\
		...\\
		&= (x_1 ~ x_2 ~ x_3 ~ ... ~ x_l)
	\end{aligned}
	$$
	
	De cette égalité on déduit la seconde, on peut facilement vérifier que pour toute transposition $(a ~ b)$ on a $\sigma(a ~ b) \sigma^{-1}=(\sigma(a) ~ \sigma(b))$, et on en déduit immédiatement
	
	$$
	\begin{aligned}
		\sigma(x_1 ~ x_2 ~...~x_l)\sigma^{-1} &= \sigma(x_1 ~ x_2)(x_2 ~ x_3)...(x_{l-1} ~ x_l)\sigma^{-1} \\
		&=\sigma(x_1 ~ x_2)\sigma^{-1} \sigma(x_2 ~ x_3)...\sigma^{-1} \sigma(x_{l-1} ~ x_l)\sigma^{-1} \\
		&= (\sigma(x_1) ~ \sigma(x_2))(\sigma(x_2) ~ \sigma(x_3))...(\sigma(x_{l-1}) ~ \sigma(x_l)) \\
		&= (\sigma(x_1) ~ \sigma(x_2) ~ \sigma(x_2) ~ ... ~ \sigma(x_l))
	\end{aligned}
	$$
	
	\cqfd
\end{myproof}

\begin{myproposition}
	Soit $c = (x_1 ~ x_2 ... x_l)$ un $l$-cycle, $c$ est d'ordre $l$.
\end{myproposition}

\begin{myproof}
	Soit $i$ tel que $1 \leqslant i \leqslant l$, on montre que pour tout $0 \leqslant k \leqslant l$ : $c^{k}(x_{1 + (i \mod l)}) = x_{1 + (i + k \mod l)}$
	
	$$
	\begin{aligned}
		c^0(x_{1 + (i \mod l)}) &= x_1 \\
		c^1(x_{1 + (i \mod l)}) &= x_{1 + (i \mod l)} = x_{1 + (i+1 \mod l)}\\
		&...\\
		c^{k+1}(x_{1 + (i \mod l)}) &= c\left(c^k(x_{1 + (i \mod l)})\right) =c(x_{1+(i+k \mod l)}) = x_{1+(i+k+1 \mod l)}\\
		&...\\
		c^{l-1}(x_{1 + (i \mod l)}) &= x_{1 + (i + l-1 \mod l)}\\
		\text{ et } c^l(x_{1 + (i \mod l)}) &= x_{1 + (i + l \mod l)} = x_{1 + (i \mod l)}
	\end{aligned}
	$$
	
	Ainsi pour tout $0 < k \leqslant l$, $c^k(x_{1 + (i \mod l)}) = x_{1 + (i+k \mod l)}$ et $1 + (i \mod l) = 1 + (i + k \mod l)$ si et seulement $k$ est un multiple de $l$, donc si $k = l$, $l$, on a ainsi montré que le plus petit $d > 0$ vérifiant pour tout $1 \leqslant k \leqslant l$ $c^d(x_k)=x_k$ est $d = l$, donc l'ordre de $c$ est $l$.
\end{myproof}

\section{Décomposition en cycles}

\subsection{Étude}

Soit $\sigma \in \mathscr{S}_n$ avec $n \geqslant n$, alors on a une action naturelle de $\langle \sigma \rangle$ sur $\{1, ...n\}$ donnée par $$\anonfunc{\langle \sigma \rangle \times \{1, ...n\}}{\{1, ...n\}}{(g, i)}{g(i)}$$

\begin{myrem}
	$\forall i \leqslant n, ~ (\langle \sigma \rangle \cdot i = \{i\} \Longleftrightarrow i \notin supp(\sigma))$
\end{myrem}

Soit $x \in \{1, ..n\}$, on supposera que $\sigma(x) \neq x$, à quoi ressemble $\langle \sigma \rangle \cdot x$ ?

On note $l := \min \{d > 0 ~ | ~ \sigma^d(x) = x\}$, il existe car cet ensemble est minoré par 1 et non-vide puisqu'il contient l'odre de $\sigma$.

\begin{myexer}
	On montre que $l ~ | ~ k$ où $k$ est l'ordre de $\sigma$, on considère la division euclidienne de $k$ par $l$ :
	
	$$k = ql + r, ~ 0 \leqslant r < l$$
	
	alors $x = \sigma^k(x) = \sigma^r(\sigma^{ql}(x)) = \sigma^r(x)$ alors par minimalité de $l$, on en déduit $r = 0$ et donc $l ~ | ~ k$.
\end{myexer}

L'orbite de $x$ sous l'action de $\langle x \rangle$ est donc l'ensemble $\{x, \sigma(x), ... \sigma^{k-1}(x)\}$, en effet $x, \sigma(x), ... \sigma^{k-1}(x)$ sont tous distincts car si $i \leqslant j < l$ vérifient $\sigma^i(x) = \sigma^j(x)$ alors $\sigma^{j - i}(x) = x$ et donc $0 \leqslant j - i < l$ et par minimalité de $l$ on déduit $j - i = 0$.

Pour conclure cette étude, on montre que les restrictions de $\sigma$ et $c=(x ~ \sigma(x) ~ \sigma^2(x) ~ ... ~ \sigma^{l-1}(x))$ à $\langle\sigma \rangle \cdot x$ sont égales :

\begin{itemize}
	\item $\sigma^i(x)  \stackrel{\sigma}{\longmapsto} \sigma^{i+1}(x)$
	\item $\sigma^i(x)  \stackrel{c}{\longmapsto} \sigma^{i+1}(x)$
\end{itemize}

\subsection{Existence}

Soit $\sigma \in \mathscr{S}_n$ telle que $\sigma \neq id$, $\omega_1, ...\omega_r$ une énumération ds orbites à au moins deux éléments et pour tout $i \leqslant r$ on pose $l_i := \text{Card}(\omega_i)$ et $x_i \in \omega_i$.

\begin{myrem}
	$supp(\sigma) = \omega_1 \sqcup \omega_2 ... \sqcup \omega_r$
\end{myrem}

On note pour tout $k$, $c_k := (x_k ~ \sigma(x_k) ~ \sigma^2(x_k) ~ ... ~ \sigma^{l-1}(x_k))$

Pour tout $i \leqslant r$, $\sigma$ et $c_i$ coïncident sur $\omega_i$, et pour tout $j \leqslant r$ tel que $i \neq j$, $c_j$ et $id$ coïncident sur $\omega_i$, donc $c_1 \circ c_2 ... \circ c_r$ coïncide avec $\sigma$ sur $\omega_i$.

Ainsi, $c_1 \circ c_2 ... \circ c_r$ coïncide avec $\sigma$ sur la réunion des orbites $\omega_1 \sqcup \omega_2 ... \sqcup \omega_r = supp(\sigma)$, de plus pour tout $x \in \{1, ...n\} \backslash supp(\sigma)$ : $\sigma(x) = x$ et $(c_1 \circ c_2 ... \circ c_r)(x) = x$ car $x \notin \omega_1, ..\omega_r$.

\subsection{Unicité}

Soit $\sigma \in \mathscr{S}_n$ telle que $\sigma \neq id$ et une décomposition en cycles à supports disjoint $\sigma = c_1 \circ c_2 ... \circ c_r$, montrons que cette décomposition est la même que l'on a construit précédemment.

\paragraph*{}
Pour commencer, montrons que les orbites à au moins deux éléments de $\sigma$ sont les supports des cycles $c_1, c_2, ...c_r$.

Soit $i \leqslant r$, $x_i \in supp(c_i)$ et $l_i$ sa longueur, alors pour tout $k \geqslant 0$ on a que $\sigma^k(x_i) = c_i^k(x_i)$ car les supports sont stables par application des cycles (pour tout $i, j$ : $c_i(supp(c_j)) = supp(c_j)$, le cas $i=j$ est une conséquence d'une remarque précédente, mais si $i \neq j$ alors chaque élément de $supp(c_j)$ est laissé fixe par $c_i$ car les supports sont disjoints) alors $\forall k \geqslant 0, ~ \sigma^k(x_i) = c_i^k(x_i)$, donc $\langle \sigma \rangle \cdot x_i = \{x_i, c(x_i), ... c^{l_i - 1}(x_i)\}$

Ainsi tous les supports $supp(c_1), supp(c_2)...supp(c_r)$ sont des orbites à au moins deux éléments de l'action de $\langle \sigma \rangle$ sur $\{1, ...n\}$, et réciproquement toute orbite à au moins deux éléments est le support d'un cycle : soit $x$ tel que $\sigma(x) \neq x$, alors nécessairement $x$ appartient à un support $supp(c_i)$, sinon il serait laissé fixe par tous les cycles et donc par $\sigma$.

\paragraph*{}
Pour finir, soient $i \leqslant r$ et $x \in supp(c_i)$, montrons que $\sigma_{| supp(c_i)} = c_{i| supp(c_i)} = (x, \sigma(x), ... \sigma^{l-1}(x))$ : pour $y \in \langle \sigma \rangle \cdot x$, il existe $k$ tel que $y = \sigma^k(x)$, alors $\sigma(y) = \sigma^{k+1}(x)$ et $\sigma(y) = (c_1 \circ c_2 .. \circ c_r)(y) = c_i(y)$ car les supports des cycles sont disjoints.

\begin{mycor}
	\leavevmode
	\begin{enumerate}
		\item Si $\sigma \neq id$, et que sa décomposition en cycles à supports disjoints est $\sigma = c_1 \circ ... c_r$ alors l'ordre de $\sigma$ est le PPCM des longueurs des cycles.
		
		\item L'ensemble des transpositions engendre $\mathscr{S}_n$
	\end{enumerate}
\end{mycor}

On rappelle, comme montré en TD, que dans un groupe $G$, si deux éléments $g$ et $g'$ sont tels que
\begin{itemize}
	\item $g$ est d'ordre fini $d$
	\item $g'$ est d'ordre fini $d'$
	\item $\langle g \rangle \cap \langle g' \rangle = \{e\}$
\end{itemize}

alors $g \star g'$ est d'ordre $PPCM(d, d')$.

\section{La signature}

Soit $n \geqslant 2$, la signature $\varepsilon$ est un homomorphisme de $\mathscr{S}_n$ dans $\{-1, 1\}$ défini par

$$\varepsilon(\sigma) = \prod_{i < j} \frac{\sigma(i) - \sigma(j)}{i - j}$$

Elle est bien définie car $$\varepsilon(\sigma) = \prod_{i < j} \frac{\sigma(j) - \sigma(i)}{j - i} = \frac{\displaystyle \prod_{i < j} \sigma(j) - \sigma(i)}{\displaystyle \prod_{i < j} j - i}$$ et en notant $\varepsilon_{i, j} \in \{-1, 1\}$ le signe de $\sigma(j) - \sigma(i)$ on peut réécrire

$$
\begin{aligned}
\varepsilon(\sigma) &= \frac{\displaystyle \prod_{i < j} \varepsilon_{i, j} |\sigma(j) - \sigma(i)|}{\displaystyle \prod_{i < j} j - i}\\
&=\prod_{i < j} \varepsilon_{i, j} \cdot \frac{\displaystyle \prod_{i < j} |\sigma(j) - \sigma(i)|}{\displaystyle \prod_{i < j} j - i} \\
&=\prod_{i < j} \varepsilon_{i, j} \cdot \frac{\displaystyle \prod_{i < j} \max(\sigma(i), \sigma(j)) - \min(\sigma(i), \sigma(j))}{\displaystyle \prod_{i < j} j - i}
\end{aligned}
$$

et en effectuant le changement de variable $(i', j') = (\min(\sigma(i), \sigma(j)), \max(\sigma(i), \sigma(j)))$ (qui est autorisé car il définit une bijection dans l'ensemble des paires ordonnées $\{(i, j) ~ | ~ 1 \leqslant i < j \leqslant n\}$) on a enfin :

$$
\begin{aligned}
\varepsilon(\sigma) &= \prod_{i < j} \varepsilon_{i, j} \cdot \frac{\displaystyle \prod_{i < j} \max(\sigma(i), \sigma(j)) - \min(\sigma(i), \sigma(j))}{\displaystyle \prod_{i < j} j - i} \\
&= \prod_{i < j} \varepsilon_{i, j} \cdot \frac{\displaystyle \prod_{i' < j'} j' - i'}{\displaystyle \prod_{i < j} j - i} \\
&=\prod_{i < j} \varepsilon_{i, j} \in \{-1, 1\}
\end{aligned}
$$

\begin{myproposition}
	Si $n\geqslant 2$, alors
	\begin{itemize}
		\item Si $\tau = (a ~ b)$ est une transposition, (on peut supposer $a < b$), alors $$\varepsilon(\tau) = \prod_{i < j} \frac{\tau(j) - \tau(i)}{j - i} = 1 \cdot 1 \cdot ... \cdot \underbrace{\frac{\tau(b) - \tau(a)}{b - a}}_{\frac{a - b}{b - a} = -1} \cdot 1 \cdot ... \cdot 1 = -1$$
		
		\item $\varepsilon$ est l'unique homomorphisme non trivial de $\mathscr{S}_n$ dans $\mathbb{C}^\times$
	\end{itemize}
\end{myproposition}

en effet $\mathscr{S}_n$ est engendré par les transpositions alors tout homomorphisme sur ce groupe est déterminé par la valeur qu'il prend sur les transpositions.
		
Toutes les transpositions sont conjuguées, alors et pour toutes permutations $\sigma, \omega$ et tout homomorphisme $\funcshort{\varphi}{\mathscr{S}_n}{\{-1, 1\}}$: $$\varphi(\sigma \circ \omega \circ \sigma^{-1}) = \varphi(\sigma)\varphi(\omega)\varphi(\sigma)^{-1}=\varphi(\omega)$$

Ainsi, si $\varphi$ vaut $1$ sur une (et donc toutes) les transpositions, alors sachant que toute permutation se décompose en produit de transpositions, $\varepsilon$ vaut $1$ sur toute permutation. Si au contraire elle vaut -1 sur une (et donc toutes) les transpositions, alors elle est égale à la signature.

\begin{mydef}
	Soit $n \geqslant 2$, on appelle \textit{$n$-ième groupe alterné} $\mathcal{A}_n$ le noyau de la signature.
\end{mydef}

\begin{myproposition}
	\leavevmode
	\begin{enumerate}
		\item Si $l \geqslant 2$ et $\sigma$ est un $l$ -cycle, alors $\varepsilon(\sigma)=(-1)^{l-1}$.
		\item Soit $\sigma \in \mathscr{S}_n$ et $\sigma = c_1 \circ ... c_r$ une décomposition en cycles à supports disjoints de longueurs $l_1, l_2, ...l_r$, alors $\varepsilon(\sigma)=(-1)^{l_1 + l-1 + ... l_r - r}$.
		
		Et en considérant chaque $l_i$ comme le cardinal d'une orbite non-ponctuelle de l'opération, sachant $n = l_1 + l_2 + ... + l_r + f$ où $f$ est le nombre de points fixes, alors $l_1 + l_2 + ... l_r - r = n - (f + r)$ et donc $\varepsilon(\sigma) = (-1)^{n - (f+r)}$ et comme $f+r = \text{Card}\left(\ClDr{[n]}{\langle \sigma \rangle}\right) := o$ alors $\varepsilon(\sigma) = (-1)^{n-o}$
		
		\item $\mathcal{A}_5$ est constitué de :
		\begin{itemize}
			\item $id$ (1 élément)
			\item doubles transpositions ($\binom{5}{2} \binom{3}{2} / 2 = 15$ éléments)
			\item de 3-cycles (20)
			\item de 5-cycles (24)
		\end{itemize}
		\item $\text{Card}(\mathcal{A}_n)  = \frac{(n-1)!}{2}$ car $\mathscr{S}_n = \mathcal{A}_n \sqcup (1 ~ 2) \circ \mathcal{A}_n$, ou bien car $\ClDr{\mathscr{S}_n}{\ker \varepsilon}$ est en bijection avec $Im(\varepsilon)$
	\end{enumerate}
\end{myproposition}

\part{Sous-groupes distingués et groupes quotient}

\section{Sous-groupes distingués}

\begin{myproposition}
	Soit $(G, \star)$ un groupe et $G \leqslant G$, les conditions suivantes sont équivalentes :
	\begin{enumerate}
		\item $\forall g \in G, ~ g H = H g$
		\item $\forall g \in G, ~ g H g^{-1} \subseteq H$
		\item $\forall g \in G, ~ g H g^{-1} = H$
	\end{enumerate}
\end{myproposition}

\begin{mydef}
	On dit que $H$ est \textit{distingué dans $G$}, noté  $H \vartriangleleft G$ si :$\forall g \in G, ~ g H g^{-1} = H$.
\end{mydef}

\begin{myexmpl}
	\leavevmode
	\begin{enumerate}
		\item Si $g$ est abélien, $H \vartriangleleft G$
		\item $\{e\} \vartriangleleft G$
		\item $Z(G) \vartriangleleft G$
	\end{enumerate}
\end{myexmpl}

\begin{myproposition}
	Si $H$ est le noyau d'un homomorphisme de groupe défini sur $G$, alors $H \vartriangleleft G$.
\end{myproposition}

En effet $\varphi$ est un homomorphisme tel que $H= \ker \varphi$, pour tout $h \in H$ et tout $g \in G$, $\varphi(g \star h \star g^{-1}) = \varphi(g)\varphi(h)\varphi(g)^{-1}=\varphi(g)\varphi(g)^{-1}=e$, donc $g H g^{-1} = H$.

\begin{myexmpl}
	\leavevmode
	\begin{itemize}
		\item $\mathcal{A}_n \vartriangleleft \mathscr{S}_n$
		\item $SL_n(\mathbb{R}) \vartriangleleft GL_n(\mathbb{R})$
		\item $SO_n(\mathbb{R}) \vartriangleleft O_n(\mathbb{R})$
		\item $SU_n(\mathbb{C}) \vartriangleleft U_n(\mathbb{C})$
	\end{itemize}
\end{myexmpl}

\section{Groupes quotients}

On note $\funcshort{\pi}{G}{\ClGa{G}{H}}$ la surjection canonique.

\begin{myproposition}
	On suppose $H \vartriangleleft G$, alors il existe une et une seule loi de composition interne $*$ sur $\ClGa{G}{H}$ telle que $\ClGa{G}{H}$ est un groupe pour cette loi et que $\pi$ soit un homomorphisme de groupes de $G$ dans $\ClGa{G}{H}$
\end{myproposition}

\begin{myexer}
	S'il existe une loi $*$ de $\ClGa{G}{H}$ vérifiant les proposition précédentes soient vraies, alors $H \vartriangleleft G$.
\end{myexer}

\begin{myrem}
	\leavevmode
	\begin{itemize}
		\item $H= \pi(e)$ est l'élément neutre de $\ClGa{G}{H}$
		\item $\ker \pi = H$
	\end{itemize}
\end{myrem}

\begin{mycor}
	$H \vartriangleleft G$ si et seulement s'il existe un homomorphisme de groupes de domaine $G$ tel que $H = \ker \varphi$
\end{mycor}

\begin{myexmpl}
	\leavevmode
	
\end{myexmpl}

\begin{myexer}
	Soit $n \in \mathbb{Z}$ il existe une application bijective $\funcshort{\varphi}{\mathbb{U}_n}{\ClGa{\mathbb{Z}}{n\mathbb{Z}}}$ telle que $\varphi(e^{\frac{2i k \pi}{n}})=\overline{k}$, c'est alors un homomorphisme de groupes et donc un isomorphisme, et sa bijection réciproque $\psi$ est définie par $\psi(\overline{k})=e^{\frac{2i k \pi}{n}}$.
\end{myexer}

\begin{myexmpl}
	\leavevmode
	\begin{itemize}
		\item Soit $n \geqslant 2$, $\mathcal{A}_n \vartriangleleft \mathscr{S}_n$ et $\ClGa{\mathscr{S}_n}{\mathcal{A}_n} \cong \ClGa{\mathbb{Z}}{n\mathbb{Z}}$
		
		\item Soit $n \geqslant 1$, $SO_n(\mathbb{R}) \vartriangleleft O_n(\mathbb{R})$ et $\ClGa{O_n(\mathbb{R})}{SO_n(\mathbb{R})} = \{\text{classes de $I_n$ et classes de $diag(-1, 1, ...1)$}\}$
	\end{itemize}
\end{myexmpl}

\begin{myexer}
	Soit $K=\{id, (1~2)(3~4),(1~3)(2~4),(1~4)(2~3)\}$ est un sous-groupe de $\mathcal{A}_4$ et c'est une réunion de classes de conjugaisons.
	
	L'ensemble quotient est de cardinal 3, donc $\ClGa{\mathcal{A}_4}{K} \cong \mathbb{U}_3$
\end{myexer}

\section{Passage au quotient des homomorphismes}

\begin{mythm}
	Soient deux groupe $(G, \star)$ et $(G', \star)$, $H \vartriangleleft G$  et $\varphi$ un homomorphisme de groupes de $G$ dans $G'$.
	
	Si $H \subseteq \ker \varphi$ alors il existe un unique homomorphisme $\funcshort{\psi}{\ClGa{G}{H}}{G'}$ tel que $\psi \circ \pi = \varphi$
\end{mythm}

\begin{myproof}
	\begin{proofpart}{Unicité}

		Soient $\funcshort{\psi, \psi'}{\ClGa{G}{H}}{G'}$ deux homomorphismes tels que $\psi \circ \pi = \psi' \circ \pi = \varphi$
		
		Soit $x \in \ClGa{G}{H}$, il existe $g \in G$ tel que $pi(g) = x$ par surjectivité de $\pi$ alors $\psi(x) = \psi(\pi(g)) = \varphi(g) $ 
		
		(TODO)
	\end{proofpart}
	
	\begin{proofpart}{}
		
		On vérifie que $\varphi$ est constante sur les classes d'équivalence de $\sim$ définie par $\forall g, g' \in G, ~ \exists h \in H ~ : ~ g' = g h$
		
		Donc $\varphi(g') = \varphi(gh)=\varphi(g)\varphi(h)=\varphi(g)$
		
		Il existe donc $\funcshort{psi}{\ClGa{G}{H}}{arG'}$ telle que $\psi \circ \pi = \varphi$.
		
		On vérifie que $\psi$ est un homomorphisme de groupes :
		
		Soient $x, x' \in \ClGa{G}{H}$ il existe $g, g' \in G$ tels que $x = \pi(g)$ et $x'=\pi(g')$
		
		Alors $x x' = \pi(g)\pi(g') = \pi(gg')$ donc $\psi(x x') = \psi(\pi(gg'))=\varphi(gg')=\varphi(g)\varphi(g')=\psi(x)\psi(x')$
	\end{proofpart}
\end{myproof}

\begin{myproperty}
	Soient deux groupes $G$ et $G'$, $H$ un sous-groupe distingué de $G$ et $f$ un homomorphisme de $G$ dans $G'$ tel que $H \subseteq \ker f$, alors il existe un unique homomorphisme $\varphi$ de $\ClGa{G}{H}$ dans $G'$ tel que $\varphi \circ \pi = f$.
	
	De plus $Im(f)=Im(\varphi)$ et $\ker \varphi = \ClGa{\ker f}{H}$.
\end{myproperty}

\begin{mycor}
	\leavevmode
	\begin{enumerate}
		\item Il existe un unique homomorphisme injectif $\funcshort{\varphi}{\ClGa{G}{\ker f}}{G'}$ tel que $\varphi \circ \pi = f$.
		
		\item Les groupes $\ClGa{G}{\ker f}$ et $Im(f)$ sont isomorphes.
		
		\item Si de plus, $f$ est surjective, alors $f$ induit un isomorphisme de groupes $\funcshort{\varphi}{\ClGa{G}{\ker f}}{G'}$ tel que $\varphi \circ \pi = f$.
	\end{enumerate}
\end{mycor}

\begin{myexer}
	Soit $g \in GL_n(\mathbb{R})$, si pour tout $u \in \mathbb{K}^n \setminus \{0\}$ on a que $gu$ est proportionnel à $u$, alors il existe $\lambda \neq 0$ tel que $g= \lambda I_n$.
\end{myexer}

\begin{mydef}
	Soit $H = \{\lambda I_n ~ | ~ \lambda \neq 0\} \leqslant GL_n(\mathbb{K})$, on note le groupe projectif linéaire
	
	$$PGL_n(\mathbb{K}) = \ClGa{GL_n(\mathbb{K})}{H}$$
\end{mydef}

\section{Un théorème d'isomorphisme}

\begin{mythm}
	Soit $G$ un groupe, $H$ et $K$ des sous-groupes distingués de $G$ tels que $K \subseteq H$, alors
	
	\begin{enumerate}
		\item $K \vartriangleleft H$
		\item $\ClGa{H}{K} \vartriangleleft \ClGa{G}{K}$
		\item Les groupes $\ClGa{\left(\ClGa{G}{K}\right)}{\left(\ClGa{H}{K}\right)}$ et $\ClGa{G}{H}$ sont isomorphes.
	\end{enumerate}
\end{mythm}

\begin{myproof}
	\leavevmode
	\begin{enumerate}
		\item $\left\{
		\begin{array}{l}
			K \vartriangleleft G \text{ donc } K \vartriangleleft H \\
			K \subseteq H
		\end{array}
		\right.$
		
		\item $\ClGa{H}{K} = \{hK ~ | ~ h \in H\} \subseteq \{gK ~ | ~ g \in G\} = \ClGa{G}{K}$
		
		On note $\pi_K$ la surjection canonique de $G$ dans $\ClGa{G}{K}$, on vérifie que $\forall \alpha \in \ClGa{G}{K}, \forall \beta \in \ClGa{H}{K}, ~ \alpha \beta \alpha^{-1} \in \ClGa{H}{K}$
		
		alors $\alpha \beta \alpha^{-1} = \pi_K(g) \pi_K(h) \pi_K(g)^{-1} = \pi_K(ghg^{-1}) \in \ClGa{H}{K}$ (car $\ClGa{H}{K} = \pi_K(H)$)
		
		Donc $\ClGa{H}{K} \vartriangleleft \ClGa{G}{K}$
		
		\item On note $\funcshort{\pi}{\ClGa{G}{K}}{\ClGa{\left(\ClGa{G}{K}\right)}{\left(\ClGa{H}{K}\right)}}$ la surjection canonique, $\pi \circ \pi_K=\funcshort{f}{G}{\ClGa{\left(\ClGa{G}{K}\right)}{\left(\ClGa{H}{K}\right)}}$
		
		$f$ est surjective comme composée d'applications surjectives.
		
		$f$ est une composée d'homomorphismes de groupes, donc c'est un homomorphisme de groupes.
		
		On vérifie que $H=\ker f$
		
		\begin{proofpart}{$H \subseteq \ker f$}
			Soit $h \in H$, alors $f(h)=\pi(\pi_K(h))$, or $\pi_K(h) \in \pi_K(H) = \ClGa{H}{K} = \ker pi$, donc $f(h) = e$.
		\end{proofpart}
		
		\begin{proofpart}{$\ker f \subseteq H$}
			Soit $g \in \ker f$, donc $e = \pi(\pi_K(g))$ et donc $\pi_K(g) \in \ker \pi = \ClGa{H}{K} = \pi_K(H)$.
			
			Ainsi, il existe $h \in $ tel que $\pi_K(g) = \pi_K(h)$, en particulier $\pi_K(g^{-1}h)=e$, donc $g^{-1}h \in \ker \pi_K = K \subseteq H$ car $h \in h$.
		\end{proofpart}
	\end{enumerate}
\end{myproof}

\begin{myexmpl}
	$G = \mathbb{Z}$, $m, n \in \mathbb{Z}$ tels que $m |n$ et $H = m\mathbb{Z}$ et $K = n \mathbb{Z}$
	
	On a $K \subseteq H$, alors $\mathbb{Z}$ tant abélien on a $H\vartriangleleft G$ et $K \vartriangleleft G$, alors
	
	$$\ClGa{\mathbb{Z}}{m\mathbb{Z}} \simeq (\ClGa{\ClGa{\mathbb{Z}}{n\mathbb{Z}})}{(\ClGa{m\mathbb{Z}}{n\mathbb{Z}})}$$
\end{myexmpl}

\part{Théorème de Sylow}

\section{Théorèmes de Sylow}

\begin{mythm}
	Soit $G$ un groupe d'ordre fini et $p$ un entier premier divisant l'ordre de $G$. Soit $H$ un sous-groupe de $G$ d'ordre une puissance de $p$ alors il existe un sous-groupe $p$-Sylow de $G$ contenant $H$.
	
	En particulier, il existe un sous-groupe $p$-Sylow dans $G$.
\end{mythm}

\begin{myrem}
 Si $H$ est un sous-groupe $p$-Sylow et si $g \in G$ alors $gHg^{-1}$est un sous-groupe $p$-Sylow de G.
 
 L'opération de $G$ sur ses sous groupes $p$-Sylow définie par $g \cdot H = gHg^{-1} = ^gH$ est une opération de groupe.
\end{myrem}

\begin{mythm}
	Soit $G$ un groupe fini d'ordre $p^\alpha m$ avec $p$ un diviseur premier de $\text{Card} G$, on note $n_p$ le nombre de sous-groupes $p$-Sylow de $G$, alors
	
	\begin{itemize}
		\item L'opération par conjugaison de $G$ sur l'ensemble de ses sous-groupes est transitive, donc $n_p | p^\alpha m$
		
		\item $n_p \equiv 1 \mod p$ et $n_p | m$
	\end{itemize}
\end{mythm}

\begin{mycor}
	\leavevmode
	\begin{itemize}
		\item Si $n_p = 1$ alors l'unique $p$-Sylow de $G$ est distingué
		\item Si au moins un sous-groupe $p$-Sylow est distingué dans $G$, alors il est l'unique $p$-Sylow de $G$
	\end{itemize}
\end{mycor}

TODO: montrer (1) à l'aide de l'action de $G$ sur ses $p$-Sylow

\begin{myexmpl}
	On considère $\mathcal{A}_4$, ses sous-groupes $2$-Sylow sont d'ordre 4, et leur nombre $n_2$ vérifie :
	$$\left\{\begin{array}{l}
		n_2 \equiv 1 [2] \\
		n_2 | 3
	\end{array}
	\right.$$
	
	donc $n_2 \in \{1, 3\}$.
	
	Si $S$ est un 2-Sylow et $\sigma \in S$, alors $\sigma$ est d'ordre 1, 2 ou 4, or les 4-cycles ne sont pas de signature pair, donc $S$ contient l'identité et les doubles transpositions.

	Il n'existe donc q'un seul 2-Sylow, c'est le sous-groupe de Klein.
\end{myexmpl}

\begin{myrem}
	Si un groupe $H$ est une réunion de classes de conjugaison de $G$, alors il est distingué.
\end{myrem}

\begin{myexmpl}
	On considère $\mathcal{A}_4$, ses sous-groupes $3$-Sylow sont d'ordre 3, et leur nombre $n_3$ vérifie :
		$$\left\{\begin{array}{l}
			n_3 \equiv 1 [3] \\
			n_2 | 4
		\end{array}
		\right.$$
		
	donc $n_3 \in \{1, 4\}$.
	
	Or $\langle(1 ~ 2 ~ 3)\rangle$ et $\langle(1 ~ 2 ~ 4)\rangle$ sont des 2-Sylow, donc $n_3 = 4$.
\end{myexmpl}

\begin{myexmpl}
	On considère $\mathscr{S}_4$ et on pose $n_2$ le nombre de sous-groupes 2-Sylow; $n_2 \in \{1, 3\}$ les sous-groupes 2-Sylow sont d'ordre 8.
	
	Dans un sous-groupe d'ordre 8 on peut trouver :
	\begin{itemize}
		\item ordre 1 : L'identité
		\item ordre 2 : Les transpositions et les doubles-transpositions
		\item ordre 4 : Les 4-cycles
		\item ordre 8 : Aucun
	\end{itemize}
	
	Soit $S := \langle (1 ~ 2 ~ 3 ~ 4) , (1 ~ 2)(3~ 4)\rangle$, $S$ contient les 3 doubles transpositions, l'identité, $(1 ~ 2 ~ 3 ~ 4)$ et $(1 ~ 4 ~ 3 ~ 2)$, donc au moins 6 éléments.
	
	$|S|$ est un multiple de 4, donc $|S| \in \{4, 8, 12, 24\}$. On peut éliminer 4 et 24, $|S| \in \{8, 12\}$
	
	De plus, en considérant l'action de conjugaison de $\mathscr{S}_4$ sur l'ensemble de ses sous-groupes, on obtient que le stabilisateur de $H = \langle (1 ~ 2 ~ 3 ~ 4)\rangle$ est de cardinal 8.
	
	Enfin, $H = S$, en effet les générateurs de $S$ sont dans le stabilisateur, d'où $H \subseteq S$, et on conclut par un argument de cardinalité.
\end{myexmpl}

\begin{myexer}
	\begin{enumerate}
		\leavevmode
		\item Décrire tous les sous-groupes de $\mathcal{A}_4$ et pour chacun décrire son stabilisateur pour l'opération de conjugaison
		\item De même pour $S_4$.
	\end{enumerate}
\end{myexer}

\part{Solutions des exercices}

\paragraph{Solution de l'exercice 1}

Commençons par montrer pour tout $n > 0$, $( g^n )^{-1} = g^{-n}$ :

$$\left( g^n \right)^{-1} = (g * g^{n-1})^{-1} = ((g^{n-1})^{-1}*g^{-1})^{-1}$$

$$\left( g^n \right)^{-1} = ((g^{n-2})^{-1}*g^{-1}*g^{-1})^{-1}$$

$$\cdots$$

$$\left( g^n \right)^{-1} = \underbrace{g^{-1}*g^{-1} \dots g^{-1}}_{n \text{ fois}} = (g^{-1})^n = g^{-n}$$

Pour tout $m, n \in \mathbb{Z}$, on distingue plusieurs cas :
\begin{itemize}
	\item $m = 0$ ou $n = 0$ \checkmark
	\item $m, n > 0$ : \checkmark
	\item $m > 0, n < 0$ avec $m + n < 0$ : $$g^m * g^n = g^m * \left(g^{-1}\right)^{|n|} = g^m*\left(g^{-1}\right)^m*\left(g^{-1}\right)^{|n| - m} = e * \left(g^{-1}\right)^{|n|-m}=\left(g^{-1}\right)^{-n-m}=g^{m+n}$$
	\item $m, n < 0$ : $$g^{m+n}=\left(g^{-1}\right)^{|m|+|n|}=\left(g^{-1}\right)^{|m|}*\left(g^{-1}\right)^{|n|}=g^m*g^n$$
	\item les autres cas se démontrent de la même façon
\end{itemize}

\paragraph{Solution de l'exercice 2}

Supposons par l'absurde que $\left(\mathbb{Z}^G, \otimes\right)$ est un groupe :

\subparagraph{Stabilité de l'opération :} \checkmark

\subparagraph{Élément neutre :} On cherche $\funcshort{\epsilon}{G}{\mathbb{Z}}$ tel que
$$\forall f \in \mathbb{Z}^G, ~ \forall g \in G, ~ \sum_{h \in G}\epsilon(h)f(h^{-1}*g)=\sum_{h \in G}f(h)\epsilon(h^{-1}*g)=f(g)$$

Pour $f$ valant $1$ sur $G$ on a
$$\sum_{h \in G}\epsilon(h)=\sum_{h \in G}\epsilon(h^{-1}*g)=1$$

Vérifions que si $\epsilon$ est définie par $\epsilon(g) = \left\{
\begin{array}{l}
	1, \text{ si } g = e \\
	0, \text{ sinon}
\end{array}
\right.$, alors elle est neutre pour $\otimes$ :

$$\sum_{h \in G}\underbrace{\epsilon(h)}_{1 \text{ \textit{ssi} } h = e}f(h^{-1}*g) = f(e^{-1}*g) = f(g)$$

$$\sum_{h \in G}f(h)\underbrace{\epsilon(h^{-1}*g)}_{1 \text{ ssi } h = g}=f(g)$$

\checked

\subparagraph{Existence d'un inverse :}
Soit $\funcshort{f}{G}{\mathbb{Z}}$, il existe $\funcshort{\varphi}{G}{\mathbb{Z}}$ telle que $f \otimes \varphi = \varphi \otimes f = \epsilon$

$$\forall g \neq e, ~ \sum_{h \in G}\varphi(h)f(h^{-1}*g)=\sum_{h \in G}f(h)\varphi(h^{-1}*g)=0$$

et

$$\sum_{h \in G}\varphi(h)f(h^{-1})=\sum_{h \in G}f(h)\varphi(h^{-1})=1$$

la deuxième égalité est impossible lorsque $f$ est la fonction nulle, $\left(\mathbb{Z}^G, \otimes\right)$ n'est donc pas un groupe.

\paragraph{Solution de l'exercice 3}
Soit $K$ un sous-groupe vérifiant les propriétés suivantes :

$$
	\begin{array}{lc}
		(1) & \forall H \leqslant G, ~ A \subseteq H \Longrightarrow K \subseteq H \\
		(2) & A \subseteq K \leqslant G
	\end{array}
$$

On rappelle que 

$$
	\begin{array}{lc}
		(3) & \forall H \leqslant G, ~ A \subseteq H \Longrightarrow \langle A \rangle \subseteq H \\
		(4) & A \subseteq \langle A \rangle \leqslant G
	\end{array}
$$

$A \subseteq K$ alors d'après (3) $\langle A \rangle \subseteq K$ et $A \subseteq \langle A \rangle$ alors d'après (1) $K \subseteq \langle A \rangle$

\paragraph{Solution de l'exercice 4}
On pose $A = \{g^n ~ | ~ n \geqslant 0\}$.

$g \in A$ donc $\langle g \rangle \subseteq A$, de plus $g \in \langle g \rangle$ alors par récurrence $\forall n \geqslant 0, ~ g^n \in \langle g \rangle$, d'où $A \subset \langle g \rangle$.

\paragraph{Solution de l'exercice 6}

Soit $d > 0$ et $g \in G$, montrons l'équivalence entre les deux propositions suivantes

$$
\begin{array}{ll}
	(i) & d \text{ est l'odre de } g \\
	(ii) & g^d=e \text{ et } \forall k | d, ~ (k < d \Longrightarrow g^k \neq e)
\end{array}
$$

\noindent
\begin{proofpart}{$(i) \Longrightarrow (ii)$}

	$d$ étant l'ordre de $g$, on a $g^d=e$, et par minimalité de $d$ on a pour tout $k < d$, $g^k \neq e$ (en particulier pour tout diviseur strict de $d$).
\end{proofpart}

\begin{proofpart}{$(ii) \Longrightarrow (i)$}

	$d$ vérifie :
	\begin{enumerate}
		\item $g^d=e$
		\item $\forall k < d, ~ (k | d \Longrightarrow g^k \neq e)$
	\end{enumerate}
	
	On a que $d \geqslant ord(g)$, par minimalité de $ord(g)$.
	
	Supposons maintenant que $d \neq ord(g)$, c'est à dire que $d > ord(g)$, l'ordre de $g$ divise nécessairement $d$, d'où l'existence d'un entier $n > 1$ tel que $d= n \cdot ord(g)$.
	
	$ord(g)$ est donc un diviseur strict de $d$ ! $d$ est ainsi égal à l'ordre de $g$, sinon on aurait d'après (2) $g^{ord(g)} \neq e$
\end{proofpart}

\paragraph{Solution de l'exercice 7}

Soit $\func{f}{\mathbb{U}_n}{\mathbb{U}_n}{x}{x^d}$

\subparagraph{Noyau}

$\ker f = \{x \in \mathbb{U}_n ~ | ~ x^d = e \} = \mathbb{U}_d$

$Im(f)=\{x^d ~|~ x \in \mathbb{U}_n\}=\mathbb{U}_{\frac{n}{n \land d}}$

\paragraph{Solution de l'exercice 8}

On considère l'action des applications linéaires inversibles de $E$ sur les formes quadratiques de $E$ définie par $g \cdot q = q \circ g^{-1}$, montrons que c'est une action de groupe.

\begin{proofpart}{Composition}

	Soient $f, g \in GL(E)$ et une forme quadratique $q$ :
	
	$f\cdot(g \cdot q)=(g \cdot q) \circ f^{-1}=q\circ g^{-1}\circ f^{-1}=q \circ (f \circ g)^{-1}=(f \circ g) \cdot q$
\end{proofpart}

\begin{proofpart}{Élément neutre}

	$id \cdot q = q \circ id^{-1} = q$
\end{proofpart}

\paragraph{\boldmath $C_m \times C_n \cong C_{mn}$} Soient $m$ et $n$ premiers entre eux.

On considère l'application
$$\func{\varphi}{C_m \times C_n}{C_{mn}}{(g,h)}{gh}$$

Pour tout $(g, h), (g', h') \in C_m \times C_n$, on a :  $$\varphi((g,h)(g',h')=\varphi(gg',hh')=gg'hh'=(gh)(g'h')=\varphi(g,h)\varphi(g',h')$$

Donc $\varphi$ est un morphisme, de plus si elle est injective alors elle sera bijective car $|C_{mn}| = |C_m \times C_n|$.

Soit $(g,h) \in C_m \times C_n$ tel que $\varphi(g, h) = gh = e$

$g \in C_m$ donc l'ordre de $g$ divise $m$, de même l'ordre de $h^{-1}$ divise $n$. On a donc que l'ordre de $g=h^{-1}$ divise $m$ et $n$, or $m$ et $n$ sont premiers entre eux, alors l'ordre de $g$ divise $m \land n = 1$.

Ainsi $g = h = e$, $\varphi$ est donc injective et donc un isomorphisme.

\paragraph{Solution de l'exercice 10}

Montrons que pour tout $g \in G$ il existe un unique couple $(\omega, h) \in G / \text{Stab}(x) \times \text{Stab}(x)$ tel que $g_\omega \star h = g$, c'est-à-dire qu'il existe une bijection entre les ensembles $G$ et $G/\text{Stab}(x) \times \text{Stab}(x)$.

On pose $$\func{\varphi}{G/\text{Stab}(x) \times \text{Stab}(x)}{G}{(\omega, h)}{g_\omega \star h}$$

Pour tout $g \in G$, $g \in \omega = \text{Stab}(x)$ et il existe un certain $g_\omega \in \omega$ tel que $g_\omega \text{Stab}(x) = g \text{Stab}(x)$, c'est-à-dire qu'il existe un $h \in \text{Stab}(x)$ tel que $g_\omega \star h = g$, d'où la surjectivité de $\varphi$.

$\varphi$ est bijective car $\left|G / \text{Stab}(x) \times \text{Stab}(x)\right| = \frac{|G|}{\text{Stab}(x)} \cdot |\text{Stab}(x)| = |G|$.

\paragraph{Solution de l'exercice 11}

$\varphi$ est non-nulle alors il existe $u_0 \in V$ tel que $\varphi(u_0) \neq 0$, on pose $u_1 = \frac{u_0}{\varphi(u_0)}$ afin d'avoir $\varphi(u_1) = 1$.

Montrons $\mathcal{F} = \{M \in V\ ~ | ~ \varphi(M) = 1\} = u_1 + \ker \varphi$.

$u_1 + \ker \varphi = \{w = u_1 + v ~ | ~ v \in \ker f\}$

$u_1 + \ker \varphi = \{w = u_1 + v ~ | ~  \varphi(v) = 0\}$

$u_1 + \ker \varphi = \{w = u_1 + v ~ | ~ \varphi(w - u_1) = 0\}$

$u_1 + \ker \varphi = \{w \in V ~ | ~ \varphi(w - u_1) = 0\}$

$u_1 + \ker \varphi = \{w \in V ~ | ~ \varphi(w) = \varphi(u_1) = 1\}$

$u_1 + \ker \varphi = \mathcal{F}$

\paragraph{Solution de l'exercice \ref{exo12}}

Soient $X, Y \in \mathcal{E}$

$$u \in F \Longleftrightarrow X + u \in X + F$$
$$u \in F \Longleftrightarrow X + u \in Y + G$$
$$u \in F \Longleftrightarrow \exists v \in G ~ : ~ X + u = Y + v$$
$$u \in F \Longleftrightarrow \exists v \in G ~ : ~ Y + (v - u) = X \in X + F = Y + G$$
$$u \in F \Longleftrightarrow (v - u) \in G$$
$$u\in F \Longleftrightarrow u \in G$$
\end{document}