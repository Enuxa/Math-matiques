\documentclass[]{article}
\usepackage[utf8]{inputenc}
\usepackage{pdfpages}
\usepackage{amsmath}
\usepackage{amssymb}
\usepackage{graphicx}
\usepackage{geometry}
\usepackage{enumitem}
\usepackage{amsthm}

\geometry{hmargin=2cm}

\title{Algèbre et géométrie 1}

\author{Patrick Le Meur et Pierre Gervais}

% Environnement type théorème
\newtheorem{mythm}{Théorème}
\newtheorem{myproposition}{Proposition}
\newtheorem{myproperty}{Propriété}
\newtheorem{mylemma}{Lemme}
\newtheorem{mycor}{Corollaire}

% Environnement type texte
\theoremstyle{remark}
\newtheorem{mynot}{Notation}
\newtheorem{myrem}{Remarque}
\newtheorem{myexer}{Exercice}
\newtheorem{myproof}{Preuve}
\newtheorem{myexmpl}{Exemple}

% Environnement de définition
\theoremstyle{definition}
\newtheorem{mydef}{Définition}

\setlist[itemize]{label=-}

% Carré de fin de preuve
\newcommand{\cqfd}{
	\hfill$\square$
}

% "Checkmark" de fin d'étape de preuve
\newcommand{\checked}{
	\hfill$\checkmark$
}

% Définition de fonction
\newcommand{\func}[5]{
#1 ~ : ~ \left\{ \begin{array}{lcl}
	#2 & \longrightarrow & #3 \\
	#4 & \longmapsto & #5
\end{array}
\right.
}

\newcommand{\funcinline}[5]{
#1 ~ : ~ #2 \longrightarrow #3, ~ #4 \longmapsto #5
}

\newcommand{\funcshort}[3]{
#1 ~ : ~ #2 \longrightarrow #3
}

\newenvironment{proofpart}[1]{
	\noindent
	{\boldmath #1}
}{
	\checkmark
}

\begin{document}

\maketitle

\tableofcontents

\part{Groupes}

\section{Définitions et premiers exemples}

\begin{mydef}
	Un \textit{groupe} est un couple $(G, *)$ où
	\begin{itemize}
		\item $G$ est un ensemble
		\item $\func{*}{G \times G}{G}{(g,h)}{g*h}$ est une loi de composition interne associative admettant un élément neutre $e$, c'est à dire tel que $\forall g \in G, g * e = e * g = g$
		\item tout élément $g$ admet un symétrique pour $*$ noté $g^{-1}$ tel que $g*g^{-1}=g^{-1}*g=e$
	\end{itemize}
\end{mydef}

\begin{myrem}
\leavevmode
\begin{itemize}
	\item L'élément neutre et le symétrique d'un élément donné est unique.
	\item Pour tout $g, h \in G$ on a $(g*h^{-1})=h^{-1}*g^{-1}$
	\item Si on a $gh=e$, alors $g=h^{-1}$
	\item Soit $g \in G$ et $n > 0$, on définit $g^n=\underbrace{g*g*g...g}_{n \text{ fois}}, ~ g^0=e, ~ g^{n+1}=g*g^n$ et $g^{-n}=\left(g^{-1}\right)^{n}$
\end{itemize}
\end{myrem}

\begin{myexer}
Montrer que pour tout $m, n \in \mathbb{Z}$ on a $g^{m+n}=g^m*g^n$ et $g^{-n}=\left(g^{-1}\right)^n$
\end{myexer}

\begin{myexmpl}
	\leavevmode
	\begin{enumerate}
		\item $G=\mathbb{Z}, *=+$
		\item Soit $E$ un espace vectoriel, $(E, +)$
		\item $(\mathbb{C}^*, \times)$ et $(\mathbb{C}, +)$
		\item Si $\mathbb{K}$ est un corps, $(\mathbb{K}, *)$

		Ces exemples sont des groupes abéliens (c'est à dire commutatifs), les suivants n'en sont pas.
		
		\item Soit $(G, \cdot)$ un groupe fini, on définit $\func{\otimes}{\mathbb{Z}^G \times \mathbb{Z}^G}{\mathbb{Z}^G}{(f_1,f_2)}{\left(g \longmapsto \displaystyle \sum_{h \in G} f_1(h)f_2(h^{-1}*g)\right)}$
		
		\begin{myexer}
			Montrer que $\mathbb{Z}^G$ muni de cette opération n'est pas un groupe mais que $\otimes$ est une loi associative.
		\end{myexer}
		\item $GL_n(\mathbb{R})$ muni de la multiplication de matrices.
	\end{enumerate}
\end{myexmpl}

\begin{myproposition}
	Soit E un ensemble non-vide, on note $\mathfrak{S}(E)$ l'ensemble des applications bijectives de $E$ dans $E$ et $(\mathfrak{S}(E), \circ)$ est un groupe.
\end{myproposition}

\section{Sous-groupe}

\begin{mydef}
	Soit $(G, *)$ un groupe, on appelle \textit{sous-groupe} de $G$ toute partie $H \subseteq G$ munie de $*$ telle que $e \in H$, $\forall (h_1,h_2) \in H^2, h_1*h_2 \in H$  et $\forall h \in H, h^{-1} \in H$.
	On note $H \leqslant G$
\end{mydef}

\begin{myexmpl}
	\begin{enumerate}
		\item Si $(G, *)$ est un groupe alors $\{e\} \leqslant G$
		\item On définit $SL_n(\mathbb{R})=\{M \in \mathcal{M}_n | \det M = 1\}$ le \textit{groupe spécial linéaire} qui est un sous-groupe de $GL_n(\mathbb{R})$
		\item On définit $\mathcal{O}_n(\mathbb{R})=\{M \in \mathcal{M}_n | \,^tMM = I_n\}$ le \textit{groupe orthogonal} qui est un sous-groupe de $GL_n(\mathbb{R})$
		\item $\mathfrak{U} = \{z \in \mathbb{C} ~ | ~ |z| = 1\} \leqslant (\mathbb{C}^*, \times)$
		\item Pour $n > 0$, $\mathfrak{U}_n = \{z \in \mathbb{C}^* ~ | ~ z^n = 1\} \leqslant \mathfrak{U} \leqslant \mathbb{C}^*$
	\end{enumerate}
\end{myexmpl}

\begin{myproposition}
	\leavevmode
	\begin{enumerate}
		\item Soit $n \in \mathbb{Z}$, $n\mathbb{Z} \leqslant \mathbb{Z}$
		\item Tout sous-groupe de $\mathbb{Z}$ est de cette forme
	\end{enumerate}
\end{myproposition}

\begin{myproof}
	\leavevmode
	\begin{enumerate}
		\item $n\mathbb{Z} \subseteq \mathbb{Z}$, $0 \in \mathbb{Z}$, $xn + yn = (x+y)n \in n\mathbb{Z}$ et $-(xn) \in n\mathbb{Z}$
		\item Soit $H \leqslant \mathbb{Z}$, si $H = \{0\}$ \checkmark
		
		Soit $n = min\{h \in H ~|~ h > 0\}$ (il existe par la propriété de la borne supérieure), montrons $H = n\mathbb{Z}$

		\begin{proofpart}{$nZ \subseteq H$}
		\end{proofpart}

		\begin{proofpart}{$nZ \subset H$}\\			
			Soit $h \in H$, on considère sa division euclidienne par n : $h = nq + r$ avec $0 \leqslant r < n$.
			$h - nq = r\in H$, et $n$ est le plus petit élément non-nul, donc $r = 0$.
		\end{proofpart}
		
		\cqfd
	\end{enumerate}
\end{myproof}

\begin{mylemma}
	Soit $G$ un groupe et $\left(H_i\right)_i \in I$ une famille de sous-groupes de $G$, alors $\displaystyle \bigcap_{i \in I}H_i \leqslant G$
\end{mylemma}

\begin{mydef}
	Soit $G$ un groupe et $A$ une partie de G, l'intersection des sous-groupes de $G$ contenant $A$ est appelée \textit{sous-groupe engendré par $A$} et notée $\left\langle A \right\rangle$.
\end{mydef}

\begin{myproperty}
	\leavevmode
	\begin{itemize}
		\item $A \subseteq \langle A \rangle \leqslant G$
		\item Si $H$ est un sous-groupe contenant $A$, alors $\langle A \rangle \subseteq H$
	\end{itemize}
\end{myproperty}

\begin{myexer}
	Montrer que $\langle A \rangle$ est l'unique sous-groupe vérifiant ces propriétés.
\end{myexer}

\begin{myproperty}
	Soit $G$ un groupe et $g \in G$, $\left\langle\{g\}\right\rangle = \langle g \rangle = \{g^n, n \in \mathbb{Z}\}$
\end{myproperty}

\begin{myexer}
	Le démontrer.
\end{myexer}

\begin{myproperty}
	Soit $A$ une partie de $G$, $\langle A \rangle$ est l'ensemble des éléments de la forme $a_1^{n_1}*a_2^{n_2}*...*a_p^{n_p}$ où $a_i \in A$ et $n_i \in \mathbb{Z}$.
\end{myproperty}

\begin{myproof}
	On pose $K$ l'ensemble des éléments de cette forme.
	Montrons
	\begin{enumerate}
		\item $A \subseteq K$ et $K \leqslant G$
		\item Pour tout $H \leqslant G$ tel que $A \subseteq H$ on a $K \subseteq H$
	\end{enumerate}
\end{myproof}

\begin{myexmpl}
	\begin{enumerate}
	\item Soient $k$ et $n$ deux entiers relatifs, $\langle k, n \rangle = k\mathbb{Z} + n\mathbb{Z} = (k \land n) \mathbb{Z}$ d'après l'identité de Bézout.
	\item Soit $n > 0$, si $M \in \mathcal{O}_n(\mathbb{R})$, alors il existe $P \in \mathcal{O}_n(\mathbb{R})$ et $r, s \in \mathbb{N}$, $\theta_1, ...\theta_p \in \mathbb{R}$ tels que $P^{-1}MP$ soit diagonale par blocs :
	
	$$
		\left(
		\begin{array}{ccccc}
			I_r &  &  &  & \\
			 & -I_s & & &\\
			 & & R(\theta_1) & &\\
			 & & & \ddots & \\
			 & & & & R(\theta_p)\\
		\end{array}
		\right)
	$$
	\end{enumerate}
\end{myexmpl}

\begin{myexer}
	Montrer que $\mathcal{O}_n(\mathbb{R})$ est engendré par les réflexions, c'est à dire les matrices orthogonales semblables à $\left(\begin{array}{cccc}
		-1 &&&\\
		&1&&\\
		&&\ddots&\\
		&&&1\\
	\end{array}\right)$ en base orthonormée.
\end{myexer}

\section{Ordre d'un élément dans un groupe}

Soit $g \in G$, on suppose qu'il existe $n > 0$ tel que $g^n=e$.
On a alors $\langle g \rangle = \{e, g, g^2, g^3, ..., g^{n-1}\}$, en effet pour tout $k > 0$, de division euclidienne $k=nq+r$ avec $0 \leqslant r < n$, on a $g^k=g^{nq+r}=e^q g^r=g^r$ d'où $g^r \in \{e, g, g^2, ..., g^{n-1}\}$.

\begin{mydef}
	Soit $g \in G$, on définit \textit{l'ordre} de $g$ par $d=\min \{k > 0 ~ | ~ g^k = e\}$, on a ainsi que $e, g, g^2, ..., g^d$ sont deux à deux distincts. On en conclut que $\langle g \rangle = \{g^k ~ | ~ 0 \leqslant k < d\}$ est de cardinal $d$.
\end{mydef}

En effet $0 \leqslant k \leqslant l < d$, on a $g^l=g^k \Longrightarrow g^{l-k}=e$.

Or $0 \leqslant l - k < d$ et par minimalité de $d$, $l=k$.

\begin{myexmpl}
	\leavevmode
	\begin{enumerate}
		\item Dans $(\mathfrak{U}, \times)$, pour $n > 0$ on a $g=\exp\left(\frac{2i \pi}{n}\right)$
		
		g est d'ordre fini égal à $n$.
		
		\item Dans $GL_n(\mathbb{R})$ $\left(\begin{array}{cc}
			1 & 0 \\
			0 & -1 \\
		\end{array}\right)$ est d'ordre 2 et $\left(\begin{array}{cc}
					0 & -1 \\
					1 & 0 \\
				\end{array}\right)$ d'ordre 4.
	\end{enumerate}
\end{myexmpl}

\begin{mythm}
	Soit $G$ un groupe fini et $H \leqslant G$, alors $|H|$ divise  $|G|$. 
\end{mythm}

\begin{mycor}
	Soit $g$ un élément d'un groupe fini, $g$ est d'ordre fini divisant $|G|$.
\end{mycor}

\begin{myexmpl}
	Dans $\mathfrak{U}_6$, d'ordre 6, les éléments peuvent avoir pour ordre 1, 2, 3 et 6.
\end{myexmpl}

\begin{myproperty}
	$d > 0$ est l'ordre de $g$ si et seulement si $g^d=e$ et pour tout diviseur strict $k$ de $d$ on a $g^k \neq e$.
\end{myproperty}

\begin{myexer}
	Le démontrer.
\end{myexer}

\begin{myrem}
	Si $g \in G$ et $p$ est un nombre premier tel que $g^p=e$, alors $g=e$ ou l'ordre de $g$ est $p$.
\end{myrem}

\section{Homomorphisme de groupe}

\subsection{Définition}

\begin{mydef}
	Soient $G$ et $G'$ deux groupes, un homomorphisme de $G$ dans $G'$ est une application de $G$ dans $G'$ tel que .
\end{mydef}

\begin{myexmpl}
	\leavevmode
	\begin{enumerate}
		\item On considère $\funcinline{\phi}{(\mathbb{R}, +)}{(\mathbb{R}^*_+, \times)}{x}{\exp(x)}$
		
		On a $\forall x, y \in \mathbb{R}, ~ \exp(x+y)=\exp(x)\exp(y)$
		
		$\ln$ est l'application réciproque.
		
		\item Le déterminant est un homomorphisme de $(GL_n(\mathbb{R}), \times)$ dans $(\mathbb{R}^*, \times)$
	\end{enumerate}
\end{myexmpl}

\begin{myrem}
	Un homomorphisme $f$ vérifie 
	\begin{itemize}
		\item $f(e)=e'$, car $f(e)=f(e \cdot e)=f(e)f(e)$ puis en simplifiant : $e' = f(e)$
		\item Pour tout $g \in G$, $f(g^{-1})=f(g)^{-1}$
	\end{itemize}
\end{myrem}

\subsection{Étude des homomorphismes de $\mathbb{Z}$}

Soit $(G, \star)$ un groupe et un homomorphisme  $\funcshort{f}{(\mathbb{Z}, +)}{(G, \star)}$, on a :
\begin{itemize}
	\item $f(1) \in G$
	\item $\displaystyle \forall n > 0, ~ f(n) = f\left(\sum_{i=1}^{n} 1\right) = \prod_{i=1}^{n} f(1) = f(1)^n$
	
	et $\displaystyle \forall n > 0, ~ f(-n) = f(n)^{-1} = \left(f(1)^n\right)^{-1} = f(1)^{-n}$
\end{itemize}

On en déduit immédiatement que pour tout homomorphismes $\funcshort{f_1, f_2}{(\mathbb{Z}, +)}{(G, \star)}$

$$f_1=f_2 \Longleftrightarrow f_1(1)=f_2(1)$$

\begin{mythm}
	Soit $(G, \star)$ un groupe, l'application
	$$
	\func{\phi}{\mathcal{H}(\mathbb{Z}, G)}{G}{f}{f(1)}
	$$
	est bijective et d'application réciproque
	
	$$
	\func{\phi^{-1}}{G}{\mathcal{H}(\mathbb{Z}, G)}{g}{n \longmapsto g^n}
	$$
\end{mythm}

\subsection{Compositions et isomorphismes}

\begin{mydef}
	\leavevmode
	\begin{itemize}
		\item Un homomorphisme bijectif est appelé \textit{isomorphisme}.
		\item Deux groupes sont dits \textit{isomorphes} si et seulement s'il existe un isomorphisme entre eux.
		\item Un endomorphisme bijectif est un \textit{automorphisme}.
	\end{itemize}
\end{mydef}

\begin{myproperty}
	\leavevmode
	\begin{itemize}
		\item La composée de deux homomorphismes est un homomorphismes.
		\item Si un homomorphisme est bijectif, alors son application réciproque est un homomorphisme.
		\item Soit $G$ un groupe, $Aut(G) \leqslant \mathfrak{S}_G$.
	\end{itemize}
\end{myproperty}

\begin{myexmpl}
	\leavevmode
	\begin{enumerate}
		\item
		$\left\{\begin{array}{rcl}
			\{\pm 1\} & \longrightarrow & Aut(\mathbb{Z}) \\
			1 & \longmapsto & id_{\mathbb{Z}} \\
			-1 & \longmapsto & -id_{\mathbb{Z}}
		\end{array}\right.$
	
	\item Soit $k > 0$, 
		$\left\{\begin{array}{rcl}
			\mathbb{Z} & \longrightarrow & \mathbb{Z} \\
			n & \longmapsto & kn \\
		\end{array}\right.$ est un endomorphisme mais pas un automorphisme.
	\end{enumerate}
\end{myexmpl}

\subsection{Sous-groupes associés à un homomorphisme}

\begin{myproposition}
	Soit $\funcshort{f}{G}{H}$ un homomorphisme de groupes
	\begin{itemize}
		\item Pour tout groupe $H' \leqslant H$, on a $f^{-1}(H') \leqslant G$
		
		\item Pour tout groupe $G' \leqslant G$, on a $f(G') \leqslant H$
	\end{itemize}
\end{myproposition}

\begin{mydef}
	Pour tout homomorphisme $f$ d'un groupe $G$ dans un autre $G'$, on appelle \textit{noyau de $f$} l'ensemble $\ker(f)=\{g \in G ~ | ~ f(g) = e\} \leqslant G$ et l'\textit{image de $f$} l'ensemble $Im(f)=f(G)$
\end{mydef}

\begin{myexer}
	Soient $d, n > 0$, déterminer l'image et le noyau de l'homomorphisme :
	
	$$\left\{
		\begin{array}{ccc}
			\mathfrak{U}_n & \longrightarrow & \mathfrak{U}_n \\
			x & \longmapsto & x^d
		\end{array}
	\right.$$
\end{myexer}

\begin{mythm}
	Soit $\funcshort{f}{G}{H}$ un homomorphisme de groupes, l'application
	
	$$\phi ~ : ~ \left\{\begin{array}{rcl}
		\{\text{Sous-groupes de } Im(f)\} & \longrightarrow & \{\text{Sous-groupes de } G \text{ contenant }\ker(f)\} \\
		H & \longmapsto & f^{-1}(H')
	\end{array}\right.$$
	
	est une bijection.
\end{mythm}

\begin{myproof}
	Pour tout $G' \leqslant G$ contenant $\ker f$, on définit $\theta$ par $\theta(G')=f(G') \leqslant Im(f)$.
	
	On démontre que $\theta$ et l'application réciproque de $\phi$.
	
	\begin{enumerate}
		\item Soit $H' \leqslant Im(f)$, on vérifie $f(f^{-1}(H))=H'$ car $G \longrightarrow Im(f), ~ x \longmapsto f(x)$ est surjective.
		
		\item Soit $G' \leqslant G$ tel que $\ker f \subseteq G'$, on a bien $f^{-1}(f(G'))=G'$.
	\end{enumerate}
	
	\cqfd
\end{myproof}

\part{Opérations de groupes}
\part{Groupes symétriques}
\part{Sous-groupes distingués et groupes quotient}
\part{Théorème de Sylow}

\part{Solutions des exercices}

\paragraph{Solution de l'exercice 1}

Commençons par montrer pour tout $n > 0$, $( g^n )^{-1} = g^{-n}$ :

$$\left( g^n \right)^{-1} = (g * g^{n-1})^{-1} = ((g^{n-1})^{-1}*g^{-1})^{-1}$$

$$\left( g^n \right)^{-1} = ((g^{n-2})^{-1}*g^{-1}*g^{-1})^{-1}$$

$$\cdots$$

$$\left( g^n \right)^{-1} = \underbrace{g^{-1}*g^{-1} ... g^{-1}}_{n \text{ fois}} = (g^{-1})^n = g^{-n}$$

Pour tout $m, n \in \mathbb{Z}$, on distingue plusieurs cas :
\begin{itemize}
	\item $m = 0$ ou $n = 0$ \checkmark
	\item $m, n > 0$ : \checkmark
	\item $m > 0, n < 0$ avec $m + n < 0$ : $$g^m * g^n = g^m * \left(g^{-1}\right)^{|n|} = g^m*\left(g^{-1}\right)^m*\left(g^{-1}\right)^{|n| - m} = e * \left(g^{-1}\right)^{|n|-m}=\left(g^{-1}\right)^{-n-m}=g^{m+n}$$
	\item $m, n < 0$ : $$g^{m+n}=\left(g^{-1}\right)^{|m|+|n|}=\left(g^{-1}\right)^{|m|}*\left(g^{-1}\right)^{|n|}=g^m*g^n$$
	\item les autres cas se démontrent de la même façon
\end{itemize}

\paragraph{Solution de l'exercice 2}

Supposons par l'absurde que $\left(\mathbb{Z}^G, \otimes\right)$ est un groupe :

\subparagraph{Stabilité de l'opération :} \checkmark

\subparagraph{Élément neutre :} On cherche $\funcshort{\epsilon}{G}{\mathbb{Z}}$ tel que
$$\forall f \in \mathbb{Z}^G, ~ \forall g \in G, ~ \sum_{h \in G}\epsilon(h)f(h^{-1}*g)=\sum_{h \in G}f(h)\epsilon(h^{-1}*g)=f(g)$$

Pour $f$ valant $1$ sur $G$ on a
$$\sum_{h \in G}\epsilon(h)=\sum_{h \in G}\epsilon(h^{-1}*g)=1$$

Vérifions que si $\epsilon$ est définie par $\epsilon(g) = \left\{
\begin{array}{l}
	1, \text{ si } g = e \\
	0, \text{ sinon}
\end{array}
\right.$, alors elle est neutre pour $\otimes$ :

$$\sum_{h \in G}\underbrace{\epsilon(h)}_{1 \text{ \textit{ssi} } h = e}f(h^{-1}*g) = f(e^{-1}*g) = f(g)$$

$$\sum_{h \in G}f(h)\underbrace{\epsilon(h^{-1}*g)}_{1 \text{ ssi } h = g}=f(g)$$

\checked

\subparagraph{Existence d'un inverse :}
Soit $\funcshort{f}{G}{\mathbb{Z}}$, il existe $\funcshort{\phi}{G}{\mathbb{Z}}$ telle que $f \otimes \phi = \phi \otimes f = \epsilon$

$$\forall g \neq e, ~ \sum_{h \in G}\phi(h)f(h^{-1}*g)=\sum_{h \in G}f(h)\phi(h^{-1}*g)=0$$

et

$$\sum_{h \in G}\phi(h)f(h^{-1})=\sum_{h \in G}f(h)\phi(h^{-1})=1$$

la deuxième égalité est impossible lorsque $f$ est la fonction nulle, $\left(\mathbb{Z}^G, \otimes\right)$ n'est donc pas un groupe.

\paragraph{Solution de l'exercice 3}
Soit $K$ un sous-groupe vérifiant les propriétés suivantes :

$$
	\begin{array}{lc}
		(1) & \forall H \leqslant G, ~ A \subseteq H \Longrightarrow K \subseteq H \\
		(2) & A \subseteq K \leqslant G
	\end{array}
$$

On rappelle que 

$$
	\begin{array}{lc}
		(3) & \forall H \leqslant G, ~ A \subseteq H \Longrightarrow \langle A \rangle \subseteq H \\
		(4) & A \subseteq \langle A \rangle \leqslant G
	\end{array}
$$

$A \subseteq K$ alors d'après (3) $\langle A \rangle \subseteq K$ et $A \subseteq \langle A \rangle$ alors d'après (1) $K \subseteq \langle A \rangle$

\paragraph{Solution de l'exercice 4}
On pose $A = \{g^n ~ | ~ n \geqslant 0\}$.

$g \in A$ donc $\langle g \rangle \subseteq A$, de plus $g \in \langle g \rangle$ alors par récurrence $\forall n \geqslant 0, ~ g^n \in \langle g \rangle$, d'où $A \subset \langle g \rangle$.

\paragraph{Solution de l'exercice 6}

Soit $d > 0$ et $g \in G$, montrons l'équivalence entre les deux propositions suivantes

$$
\begin{array}{ll}
	(i) & d \text{ est l'odre de } g \\
	(ii) & g^d=e \text{ et } \forall k | d, ~ (k < d \Longrightarrow g^k \neq e)
\end{array}
$$

\noindent
\begin{proofpart}{$(i) \Longrightarrow (ii)$}

	$d$ étant l'ordre de $g$, on a $g^d=e$, et par minimalité de $d$ on a pour tout $k < d$, $g^k \neq e$ (en particulier pour tout diviseur strict de $d$).
\end{proofpart}

\begin{proofpart}{$(ii) \Longrightarrow (i)$}

	$d$ vérifie :
	\begin{enumerate}
		\item $g^d=e$
		\item $\forall k < d, ~ (k | d \Longrightarrow g^k \neq e)$
	\end{enumerate}
	
	On a que $d \geqslant ord(g)$, par minimalité de $ord(g)$.
	
	Supposons maintenant que $d \neq ord(g)$, c'est à dire que $d > ord(g)$, l'ordre de $g$ divise nécessairement $d$, d'où l'existence d'un entier $n > 1$ tel que $d= n \cdot ord(g)$.
	
	$ord(g)$ est donc un diviseur strict de $d$ ! $d$ est ainsi égal à l'ordre de $g$, sinon on aurait d'après (2) $g^{ord(g)} \neq e$
\end{proofpart}

\paragraph{Solution de l'exercice 7}

Soit $\func{f}{\mathfrak{U}_n}{\mathfrak{U}_n}{x}{x^d}$

\subparagraph{Noyau}

$\ker f = \{x \in \mathfrak{U}_n ~ | ~ x^d = e \} = \mathfrak{U}_d$

$Im(f)=\{x^d ~|~ x \in \mathfrak{U}_n\}=\mathfrak{U}_{\frac{n}{n \land d}}$

\end{document}