\documentclass[]{article}
\usepackage[utf8]{inputenc}
\usepackage{pdfpages}
\usepackage{amsmath}
\usepackage{amssymb}
\usepackage{graphicx}
\usepackage{geometry}
\usepackage{enumitem}
\usepackage{amsthm}

\geometry{hmargin=2cm}

\title{Algèbre et géométrie 1}

\author{Patrick Le Meur et Pierre Gervais}

% Environnement type théorème
\newtheorem{mythm}{Théorème}
\newtheorem{myproposition}{Proposition}
\newtheorem{myproperty}{Propriété}
\newtheorem{mylemma}{Lemme}
\newtheorem{mycor}{Corollaire}

% Environnement type texte
\theoremstyle{remark}
\newtheorem{mynot}{Notation}
\newtheorem{myrem}{Remarque}
\newtheorem{myexer}{Exercice}
\newtheorem{myproof}{Preuve}
\newtheorem{myexmpl}{Exemple}

% Environnement de définition
\theoremstyle{definition}
\newtheorem{mydef}{Définition}

\setlist[itemize]{label=-}

% Carré de fin de preuve
\newcommand{\cqfd}{
	\hfill$\square$
}

% "Checkmark" de fin d'étape de preuve
\newcommand{\checked}{
	\hfill$\checkmark$
}

% Définition de fonction
\newcommand{\func}[5]{
#1 ~ : ~ \left\{ \begin{array}{lcl}
	#2 & \longrightarrow & #3 \\
	#4 & \longmapsto & #5
\end{array}
\right.
}

\newcommand{\funcinline}[5]{
#1 ~ : ~ #2 \longrightarrow #3, ~ #4 \longmapsto #5
}

\newcommand{\funcshort}[3]{
#1 ~ : ~ #2 \longrightarrow #3
}

\newenvironment{proofpart}[1]{
	\noindent
	{\boldmath #1}
}{
	\checkmark
}

\begin{document}

\maketitle

\tableofcontents

\part{Groupes}

\section{Définitions et premiers exemples}

\begin{mydef}
	Un \textit{groupe} est un couple $(G, *)$ où
	\begin{itemize}
		\item $G$ est un ensemble
		\item $\func{*}{G \times G}{G}{(g,h)}{g*h}$ est une loi de composition interne associative admettant un élément neutre $e$, c'est à dire tel que $\forall g \in G, g * e = e * g = g$
		\item tout élément $g$ admet un symétrique pour $*$ noté $g^{-1}$ tel que $g*g^{-1}=g^{-1}*g=e$
	\end{itemize}
\end{mydef}

\begin{myrem}
\leavevmode
\begin{itemize}
	\item L'élément neutre et le symétrique d'un élément donné est unique.
	\item Pour tout $g, h \in G$ on a $(g*h^{-1})=h^{-1}*g^{-1}$
	\item Si on a $gh=e$, alors $g=h^{-1}$
	\item Soit $g \in G$ et $n > 0$, on définit $g^n=\underbrace{g*g*g...g}_{n \text{ fois}}, ~ g^0=e, ~ g^{n+1}=g*g^n$ et $g^{-n}=\left(g^{-1}\right)^{n}$
\end{itemize}
\end{myrem}

\begin{myexer}
Montrer que pour tout $m, n \in \mathbb{Z}$ on a $g^{m+n}=g^m*g^n$ et $g^{-n}=\left(g^{-1}\right)^n$
\end{myexer}

\begin{myexmpl}
	\leavevmode
	\begin{enumerate}
		\item $G=\mathbb{Z}, *=+$
		\item Soit $E$ un espace vectoriel, $(E, +)$
		\item $(\mathbb{C}^*, \times)$ et $(\mathbb{C}, +)$
		\item Si $\mathbb{K}$ est un corps, $(\mathbb{K}, *)$

		Ces exemples sont des groupes abéliens (c'est à dire commutatifs), les suivants n'en sont pas.
		
		\item Soit $(G, \cdot)$ un groupe fini, on définit $\func{\otimes}{\mathbb{Z}^G \times \mathbb{Z}^G}{\mathbb{Z}^G}{(f_1,f_2)}{\left(g \longmapsto \displaystyle \sum_{h \in G} f_1(h)f_2(h^{-1}*g)\right)}$
		
		\begin{myexer}
			Montrer que $\mathbb{Z}^G$ muni de cette opération n'est pas un groupe mais que $\otimes$ est une loi associative.
		\end{myexer}
		\item $GL_n(\mathbb{R})$ muni de la multiplication de matrices.
	\end{enumerate}
\end{myexmpl}

\begin{myproposition}
	Soit E un ensemble non-vide, on note $\mathfrak{S}(E)$ l'ensemble des applications bijectives de $E$ dans $E$ et $(\mathfrak{S}(E), \circ)$ est un groupe.
\end{myproposition}

\section{Sous-groupe}

\begin{mydef}
	Soit $(G, *)$ un groupe, on appelle \textit{sous-groupe} de $G$ toute partie $H \subseteq G$ munie de $*$ telle que $e \in H$, $\forall (h_1,h_2) \in H^2, h_1*h_2 \in H$  et $\forall h \in H, h^{-1} \in H$.
	On note $H \leqslant G$
\end{mydef}

\begin{myexmpl}
	\begin{enumerate}
		\item Si $(G, *)$ est un groupe alors $\{e\} \leqslant G$
		\item On définit $SL_n(\mathbb{R})=\{M \in \mathcal{M}_n | \det M = 1\}$ le \textit{groupe spécial linéaire} qui est un sous-groupe de $GL_n(\mathbb{R})$
		\item On définit $\mathcal{O}_n(\mathbb{R})=\{M \in \mathcal{M}_n | \,^tMM = I_n\}$ le \textit{groupe orthogonal} qui est un sous-groupe de $GL_n(\mathbb{R})$
		\item $\mathbb{U} = \{z \in \mathbb{C} ~ | ~ |z| = 1\} \leqslant (\mathbb{C}^*, \times)$
		\item Pour $n > 0$, $\mathbb{U}_n = \{z \in \mathbb{C}^* ~ | ~ z^n = 1\} \leqslant \mathbb{U} \leqslant \mathbb{C}^*$
	\end{enumerate}
\end{myexmpl}

\begin{myproposition}
	\leavevmode
	\begin{enumerate}
		\item Soit $n \in \mathbb{Z}$, $n\mathbb{Z} \leqslant \mathbb{Z}$
		\item Tout sous-groupe de $\mathbb{Z}$ est de cette forme
	\end{enumerate}
\end{myproposition}

\begin{myproof}
	\leavevmode
	\begin{enumerate}
		\item $n\mathbb{Z} \subseteq \mathbb{Z}$, $0 \in \mathbb{Z}$, $xn + yn = (x+y)n \in n\mathbb{Z}$ et $-(xn) \in n\mathbb{Z}$
		\item Soit $H \leqslant \mathbb{Z}$, si $H = \{0\}$ \checkmark
		
		Soit $n = min\{h \in H ~|~ h > 0\}$ (il existe par la propriété de la borne supérieure), montrons $H = n\mathbb{Z}$

		\begin{proofpart}{$nZ \subseteq H$}
		\end{proofpart}

		\begin{proofpart}{$nZ \subset H$}\\			
			Soit $h \in H$, on considère sa division euclidienne par n : $h = nq + r$ avec $0 \leqslant r < n$.
			$h - nq = r\in H$, et $n$ est le plus petit élément non-nul, donc $r = 0$.
		\end{proofpart}
		
		\cqfd
	\end{enumerate}
\end{myproof}

\begin{mylemma}
	Soit $G$ un groupe et $\left(H_i\right)_i \in I$ une famille de sous-groupes de $G$, alors $\displaystyle \bigcap_{i \in I}H_i \leqslant G$
\end{mylemma}

\begin{mydef}
	Soit $G$ un groupe et $A$ une partie de G, l'intersection des sous-groupes de $G$ contenant $A$ est appelée \textit{sous-groupe engendré par $A$} et notée $\left\langle A \right\rangle$.
\end{mydef}

\begin{myproperty}
	\leavevmode
	\begin{itemize}
		\item $A \subseteq \langle A \rangle \leqslant G$
		\item Si $H$ est un sous-groupe contenant $A$, alors $\langle A \rangle \subseteq H$
	\end{itemize}
\end{myproperty}

\begin{myexer}
	Montrer que $\langle A \rangle$ est l'unique sous-groupe vérifiant ces propriétés.
\end{myexer}

\begin{myproperty}
	Soit $G$ un groupe et $g \in G$, $\left\langle\{g\}\right\rangle = \langle g \rangle = \{g^n, n \in \mathbb{Z}\}$
\end{myproperty}

\begin{myexer}
	Le démontrer.
\end{myexer}

\begin{myproperty}
	Soit $A$ une partie de $G$, $\langle A \rangle$ est l'ensemble des éléments de la forme $a_1^{n_1}*a_2^{n_2}*...*a_p^{n_p}$ où $a_i \in A$ et $n_i \in \mathbb{Z}$.
\end{myproperty}

\begin{myproof}
	On pose $K$ l'ensemble des éléments de cette forme.
	Montrons
	\begin{enumerate}
		\item $A \subseteq K$ et $K \leqslant G$
		\item Pour tout $H \leqslant G$ tel que $A \subseteq H$ on a $K \subseteq H$
	\end{enumerate}
\end{myproof}

\begin{myexmpl}
	\begin{enumerate}
	\item Soient $k$ et $n$ deux entiers relatifs, $\langle k, n \rangle = k\mathbb{Z} + n\mathbb{Z} = (k \land n) \mathbb{Z}$ d'après l'identité de Bézout.
	\item Soit $n > 0$, si $M \in \mathcal{O}_n(\mathbb{R})$, alors il existe $P \in \mathcal{O}_n(\mathbb{R})$ et $r, s \in \mathbb{N}$, $\theta_1, ...\theta_p \in \mathbb{R}$ tels que $P^{-1}MP$ soit diagonale par blocs :
	
	$$
		\left(
		\begin{array}{ccccc}
			I_r &  &  &  & \\
			 & -I_s & & &\\
			 & & R(\theta_1) & &\\
			 & & & \ddots & \\
			 & & & & R(\theta_p)\\
		\end{array}
		\right)
	$$
	\end{enumerate}
\end{myexmpl}

\begin{myexer}
	Montrer que $\mathcal{O}_n(\mathbb{R})$ est engendré par les réflexions, c'est à dire les matrices orthogonales semblables à $\left(\begin{array}{cccc}
		-1 &&&\\
		&1&&\\
		&&\ddots&\\
		&&&1\\
	\end{array}\right)$ en base orthonormée.
\end{myexer}

\section{Ordre d'un élément dans un groupe}

Soit $g \in G$, on suppose qu'il existe $n > 0$ tel que $g^n=e$.
On a alors $\langle g \rangle = \{e, g, g^2, g^3, ..., g^{n-1}\}$, en effet pour tout $k > 0$, de division euclidienne $k=nq+r$ avec $0 \leqslant r < n$, on a $g^k=g^{nq+r}=e^q g^r=g^r$ d'où $g^r \in \{e, g, g^2, ..., g^{n-1}\}$.

\begin{mydef}
	Soit $g \in G$, on définit \textit{l'ordre} de $g$ par $d=\min \{k > 0 ~ | ~ g^k = e\}$, on a ainsi que $e, g, g^2, ..., g^d$ sont deux à deux distincts. On en conclut que $\langle g \rangle = \{g^k ~ | ~ 0 \leqslant k < d\}$ est de cardinal $d$.
\end{mydef}

En effet $0 \leqslant k \leqslant l < d$, on a $g^l=g^k \Longrightarrow g^{l-k}=e$.

Or $0 \leqslant l - k < d$ et par minimalité de $d$, $l=k$.

\begin{myexmpl}
	\leavevmode
	\begin{enumerate}
		\item Dans $(\mathbb{U}, \times)$, pour $n > 0$ on a $g=\exp\left(\frac{2i \pi}{n}\right)$
		
		g est d'ordre fini égal à $n$.
		
		\item Dans $GL_n(\mathbb{R})$ $\left(\begin{array}{cc}
			1 & 0 \\
			0 & -1 \\
		\end{array}\right)$ est d'ordre 2 et $\left(\begin{array}{cc}
					0 & -1 \\
					1 & 0 \\
				\end{array}\right)$ d'ordre 4.
	\end{enumerate}
\end{myexmpl}

\begin{mythm}
	Soit $G$ un groupe fini et $H \leqslant G$, alors $|H|$ divise  $|G|$. 
\end{mythm}

\begin{mycor}
	Soit $g$ un élément d'un groupe fini, $g$ est d'ordre fini divisant $|G|$.
\end{mycor}

\begin{myexmpl}
	Dans $\mathbb{U}_6$, d'ordre 6, les éléments peuvent avoir pour ordre 1, 2, 3 et 6.
\end{myexmpl}

\begin{myproperty}
	$d > 0$ est l'ordre de $g$ si et seulement si $g^d=e$ et pour tout diviseur strict $k$ de $d$ on a $g^k \neq e$.
\end{myproperty}

\begin{myexer}
	Le démontrer.
\end{myexer}

\begin{myrem}
	Si $g \in G$ et $p$ est un nombre premier tel que $g^p=e$, alors $g=e$ ou l'ordre de $g$ est $p$.
\end{myrem}

\section{Homomorphisme de groupe}

\subsection{Définition}

\begin{mydef}
	Soient $G$ et $G'$ deux groupes, un homomorphisme de $G$ dans $G'$ est une application de $G$ dans $G'$ tel que .
\end{mydef}

\begin{myexmpl}
	\leavevmode
	\begin{enumerate}
		\item On considère $\funcinline{\varphi}{(\mathbb{R}, +)}{(\mathbb{R}^*_+, \times)}{x}{\exp(x)}$
		
		On a $\forall x, y \in \mathbb{R}, ~ \exp(x+y)=\exp(x)\exp(y)$
		
		$\ln$ est l'application réciproque.
		
		\item Le déterminant est un homomorphisme de $(GL_n(\mathbb{R}), \times)$ dans $(\mathbb{R}^*, \times)$
	\end{enumerate}
\end{myexmpl}

\begin{myrem}
	Un homomorphisme $f$ vérifie 
	\begin{itemize}
		\item $f(e)=e'$, car $f(e)=f(e \cdot e)=f(e)f(e)$ puis en simplifiant : $e' = f(e)$
		\item Pour tout $g \in G$, $f(g^{-1})=f(g)^{-1}$
	\end{itemize}
\end{myrem}

\subsection{Étude des homomorphismes de $\mathbb{Z}$}

Soit $(G, \star)$ un groupe et un homomorphisme  $\funcshort{f}{(\mathbb{Z}, +)}{(G, \star)}$, on a :
\begin{itemize}
	\item $f(1) \in G$
	\item $\displaystyle \forall n > 0, ~ f(n) = f\left(\sum_{i=1}^{n} 1\right) = \prod_{i=1}^{n} f(1) = f(1)^n$
	
	et $\displaystyle \forall n > 0, ~ f(-n) = f(n)^{-1} = \left(f(1)^n\right)^{-1} = f(1)^{-n}$
\end{itemize}

On en déduit immédiatement que pour tout homomorphismes $\funcshort{f_1, f_2}{(\mathbb{Z}, +)}{(G, \star)}$

$$f_1=f_2 \Longleftrightarrow f_1(1)=f_2(1)$$

\begin{mythm}
	Soit $(G, \star)$ un groupe, l'application
	$$
	\func{\varphi}{\mathcal{H}(\mathbb{Z}, G)}{G}{f}{f(1)}
	$$
	est bijective et d'application réciproque
	
	$$
	\func{\varphi^{-1}}{G}{\mathcal{H}(\mathbb{Z}, G)}{g}{n \longmapsto g^n}
	$$
\end{mythm}

\subsection{Compositions et isomorphismes}

\begin{mydef}
	\leavevmode
	\begin{itemize}
		\item Un homomorphisme bijectif est appelé \textit{isomorphisme}.
		\item Deux groupes sont dits \textit{isomorphes} si et seulement s'il existe un isomorphisme entre eux.
		\item Un endomorphisme bijectif est un \textit{automorphisme}.
	\end{itemize}
\end{mydef}

\begin{myproperty}
	\leavevmode
	\begin{itemize}
		\item La composée de deux homomorphismes est un homomorphismes.
		\item Si un homomorphisme est bijectif, alors son application réciproque est un homomorphisme.
		\item Soit $G$ un groupe, $Aut(G) \leqslant \mathfrak{S}_G$.
	\end{itemize}
\end{myproperty}

\begin{myexmpl}
	\leavevmode
	\begin{enumerate}
		\item
		$\left\{\begin{array}{rcl}
			\{\pm 1\} & \longrightarrow & Aut(\mathbb{Z}) \\
			1 & \longmapsto & id_{\mathbb{Z}} \\
			-1 & \longmapsto & -id_{\mathbb{Z}}
		\end{array}\right.$
	
	\item Soit $k > 0$, 
		$\left\{\begin{array}{rcl}
			\mathbb{Z} & \longrightarrow & \mathbb{Z} \\
			n & \longmapsto & kn \\
		\end{array}\right.$ est un endomorphisme mais pas un automorphisme.
	\end{enumerate}
\end{myexmpl}

\subsection{Sous-groupes associés à un homomorphisme}

\begin{myproposition}
	Soit $\funcshort{f}{G}{H}$ un homomorphisme de groupes
	\begin{itemize}
		\item Pour tout groupe $H' \leqslant H$, on a $f^{-1}(H') \leqslant G$
		
		\item Pour tout groupe $G' \leqslant G$, on a $f(G') \leqslant H$
	\end{itemize}
\end{myproposition}

\begin{mydef}
	Pour tout homomorphisme $f$ d'un groupe $G$ dans un autre $G'$, on appelle \textit{noyau de $f$} l'ensemble $\ker(f)=\{g \in G ~ | ~ f(g) = e\} \leqslant G$ et l'\textit{image de $f$} l'ensemble $Im(f)=f(G)$
\end{mydef}

\begin{myexer}
	Soient $d, n > 0$, déterminer l'image et le noyau de l'homomorphisme :
	
	$$\left\{
		\begin{array}{ccc}
			\mathbb{U}_n & \longrightarrow & \mathbb{U}_n \\
			x & \longmapsto & x^d
		\end{array}
	\right.$$
\end{myexer}

\begin{mythm}
	Soit $\funcshort{f}{G}{H}$ un homomorphisme de groupes, l'application
	
	$$\varphi ~ : ~ \left\{\begin{array}{rcl}
		\{\text{Sous-groupes de } Im(f)\} & \longrightarrow & \{\text{Sous-groupes de } G \text{ contenant }\ker(f)\} \\
		H & \longmapsto & f^{-1}(H')
	\end{array}\right.$$
	
	est une bijection.
\end{mythm}

\begin{myproof}
	Pour tout $G' \leqslant G$ contenant $\ker f$, on définit $\theta$ par $\theta(G')=f(G') \leqslant Im(f)$.
	
	On démontre que $\theta$ est l'application réciproque de $\varphi$.
	
	\begin{enumerate}
		\item Soit $H' \leqslant Im(f)$, on vérifie $(\theta \circ \varphi)(H')=f(f^{-1}(H'))=H'$ car la co-restriction de $f$ à $Im(f)$ est surjective.
		
		\item Soit $G' \leqslant G$ tel que $\ker f \subseteq G'$, on a $G' \subseteq (\varphi \circ \theta)(G')=f^{-1}(f(G'))$, montrons maintenant $f^{-1}(f(G')) \subseteq G'$ :
		
		Soit $x \in f^{-1}(f(G'))$, alors $f(x) \in f(G')$ et il existe $u \in G'$ tel que $f(x)=f(u)$.
		
		On a donc : $$f(x)=f(u)$$
		$$f(xu^{-1})=e$$
		$$xu^{-1} \in \ker f \subseteq G'$$
		$$x \in \underbrace{\left(G'\right) \cdot u}_{G' \text{ car } u \in G'}=G'$$
		
		Ainsi $f^{-1}(f(G')) \subseteq G'$, et donc $(\varphi \circ \theta)(G')=G'$.
	\end{enumerate}
	
	$\theta$ est bien la bijection réciproque de $\varphi$.
	
	\cqfd
\end{myproof}

\part{Opérations de groupes}

\section{Rappels sur les relations d'équivalence}

\begin{mydef}Soit $X$ un ensemble.

	Une relation binaire $\sim$ sur $X$ est une \textit{relation d'équivalence} si :
	\begin{enumerate}
		\item $\sim$ est transitive : $\forall x, y, z \in X, ~ (x \sim y \land y \sim z \Longrightarrow x \sim z)$
		\item $\sim$ est réflexive : $\forall x \in X, ~ x \sim x$
		\item $\sim$ est symétrique : $\forall x, y \in X, ~ x \sim x$
	\end{enumerate}
	
	On appelle la \textit{classe d'équivalence de $x$} l'ensemble $\{y \in X ~ | ~ x \sim y\}$ et on note $X/\sim$ l'ensemble des classes d'équivalence que l'on appelle \textit{ensemble quotient}.
	
	L'application de $X \longrightarrow X/\sim$ qui à tout élément $x$ associe sa classe d'équivalence est appelée \textit{surjection canonique}.
\end{mydef}

\begin{myproposition}
	Pour une relation $\sim$ donnée, $X/\sim$ est une partition de $X$.
\end{myproposition}

\begin{myexmpl}
	Soit $(G, \cdot)$ un groupe.
	
	\begin{enumerate}
		\item Soit $H \leqslant G$ et $\sim$ la relation définie par $\forall g_1, g_2 \in G, ~ (g_1 \sim g_2 \Longleftrightarrow \exists h \in H ~ : ~ g_1 \cdot h=g_2)$
		
		\item $\mathcal{R}$ par $\forall g_1, g_2 \in G, ~ (g_1 \mathcal{R} g_2 \Longleftrightarrow \exists h \in H ~ g_1 = h g_2 h^{-1})$
		
		\item $\forall H, K \leqslant G, ~ (H \sim K \Longleftrightarrow \exists g \in G ~ : ~ gHg^{-1}=K)$
	\end{enumerate}
\end{myexmpl}

\begin{mythm}
	Soit $\sim$ une relation d'équivalence, sur un ensemble $X$, $Y$ un ensemble et $\funcshort{f}{X}{Y}$ une application constante sur les classes d'équivalences.
	
	Il existe alors une unique application de $\funcshort{g}{(X/\sim)}{Y}$ telle que $g\circ \pi=f$ où $\pi$ est la surjection canonique.
\end{mythm}

\section{Opérations de groupes}

\begin{mydef}
	On définit une action de groupe $(G, \star)$ sur un ensemble $X$ par la donnée d'une application $$\func{\phi}{G \times X}{X}{(g,x)}{g \cdot x}$$
	vérifiant $\forall x \in G, ~ e \cdot x = x$ et $\forall g, h \in G, ~ \forall x \in X, ~ g \cdot (h \cdot x) = (h \star g) \cdot x$
\end{mydef}

\begin{myexmpl}
	Soit $G$ un groupe.
	\begin{enumerate}
		\item Soit $H$ un sous-groupe de $G$, l'action de $H$ sur $G$ définie par $h\cdot g = hg$ est appelée action de $H$ sur $G$ par \textit{translation à gauche}.
		
		\item L'action de $G$ sur lui même définie par $h\cdot g = hgh^{-1}$ est appelée action de $G$ sur lui-même par \textit{conjugaison}.
		
		\item L'action de $G$ sur $\mathcal{S}(G)$ définie par $g \cdot H=gHg^{-1}=i_g(H)$ est l'opération de $G$ sur ses sous-groupes par \textit{conjugaison}.
	\end{enumerate}
\end{myexmpl}

\begin{myproposition}
	Soient $(G, \star)$ un groupe, $X$ un ensemble et $\funcshort{\varphi}{G \times X}{X}$.
	
	$\varphi$ définit une action de groupe si et seulement si pour tout $g \in G$, l'application $\func{\varphi_g}{G}{X}{x}{g \cdot x}$ est bijective et $g\longmapsto \varphi_g$ est un homomorphisme de $(G, \star)$ dans $(\mathfrak{S}_X, \circ)$.
\end{myproposition}

\begin{myrem}
	On en déduit $\forall g, h \in G, ~ \varphi_g \circ \varphi_h = \varphi_{g \star h}$ et $\varphi_{g^{-1}}=(\varphi_g)^{-1}$
\end{myrem}

\begin{myexmpl}
	\begin{enumerate}
		\item L'action de $GL_n(\mathbb{K})$ sur $\mathbb{K}^n$ définie par $M\cdot x=Mx$ est une action de groupe.
	
		\item L'action de $GL_n(\mathbb{K})$ sur $M_{m \times n}(\mathbb{K})$ définie par $M \cdot N=MN$ est une action de groupe.
				
		\item L'action des applications linéaires inversibles de $E$ sur les formes quadratiques de $E$ définie par $g \cdot q = q \circ g^{-1}$
		
		\begin{myexer}
			Le démontrer.
		\end{myexer}
		
		\item L'action par conjugaison des matrice inversibles sur les matrices carrées est une action de groupe.
	\end{enumerate}
\end{myexmpl}

\section{Orbites, stabilisateurs}

\begin{mydef}
	Soit $G$ un groupe opérant sur $X$ et $x \in X$, on définit :
	
	\begin{itemize}
		\item L'orbite de $x$ l'ensemble $G \cdot x=\{g\cdot x ~ | ~ g \in G\}$
		
		\item Le stabilisateur de $x$ l'ensemble $Stab_G(x)=G_x = \{g \in G ~ | ~ g \cdot x = x\}$
	\end{itemize}
\end{mydef}

\begin{myexmpl}
	Considérons l'action de $G$ sur lui-même par translation à gauche.	
	On a $G \cdot x = G$ et $G_x=\{e\}$
\end{myexmpl}

\begin{myproposition}
	La relation sur $X$ définie par :
	$$\forall x, y \in X, x \sim y \Longleftrightarrow \exists g \in G ~ : ~ g \cdot x = y$$
	est une relation d'équivalence, dite \textit{associée à l'opération de $G$ sur $X$}.
\end{myproposition}

\begin{myrem}
	Si $x \in X$, alors $G \cdot x$ est la classe d'équivalence de $x$.
	
	On rappelle aussi que si $\mathcal{R}$ est une relation d'équivalence sur un ensemble $X$, alors $X / \mathcal{R}$ est une partition de $X$.
	
	En particulier, dans le cas d'une relation d'équivalence associée à une action de groupe, $x \sim y \Longleftrightarrow G \cdot x = G \cdot y$.
\end{myrem}

On note alors l'ensemble quotient $X / \sim$ par $G \backslash X$ et on l'appelle ensemble quotient de $X$ par $G$.

\begin{myrem}
	Si on se donne une action à droite sur $X$, on note $X/G$ l'ensemble des classes d'équivalence.
\end{myrem}

\begin{mydef}
	Soit $H \leqslant G$, on note l'ensemble quotient de $G$ par l'opération de translation à gauche de $H$ :
	
	$$H \backslash G = \{Hg ~ | ~ g \in G\}$$
	
	De même on note l'ensemble quotient de $G$ par l'opération de translation à droite de $H$ :
	
	$$G / H = \{gH ~ | ~ g \in G\}$$
\end{mydef}

\begin{mydef}
	Une action de $G$ sur $X$ est dite :
	\begin{itemize}
		\item \textit{transitive} s'il n'existe qu'une seule orbite
		\item \textit{fidèle} si $\forall g \in G, ~ (\forall x \in X, ~ g \cdot x = x) \Longrightarrow g=e$
		\item \textit{libre} si $\forall x \in X, \forall g \in G, g \cdot x = x \Longrightarrow g=e$
	\end{itemize}
\end{mydef}

\begin{myrem}
	L'action est fidèle si : $$\bigcap_{x \in X} Stab(x) = \{e\}$$
	et elle est transitive si $\forall x \in X, ~ Stab(x) = \{e\}$.
\end{myrem}

\begin{myexer}
	Une opération est fidèle si et seulement si l'homomorphisme associé est injectif.
\end{myexer}

\begin{myexmpl}
	\leavevmode
	\begin{enumerate}
		\item L'action de $GL_n(\mathbb{R})$ sur  $\mathbb{R}^n$ admet deux orbites : $\{0\}$ et $\mathbb{R}^n \backslash \{0\}$
		
		\item L'action de $GL_n(\mathbb{R})$ sur  $\mathbb{R}^n \backslash \{0\}$ admet une unique orbite $\mathbb{R}^n \backslash \{0\}$ donc l'action est transitive.
		
		Elle est fidèle ($(\forall x \in \mathbb{R}^n, ~ Mx=x) \Longrightarrow M=I_n$) mais elle n'est pas libre car une matrice inversible différente de $I_n$ de vecteur propre $x$ et de valeur propre $1$ vérifie $Mx=x$.
		
		\item Soient $H \leqslant G$, l'action de $H$ sur $G$ par translation à gauche.
		
		L'action est transitive si et seulement si $G = H$.
		
		L'opération est libre car pour tout $g \in G$ et $x \in G$, si on a $g \star x=x$, alors en simplifiant par $x$ on a $g=e$. Elle est donc également fidèle.
		
		\item On considère les sommets du cube de côté $2$ l'ensemble $\mathcal{C}=\{(x_1, x_2, x_3) \in \mathbb{R}^3 ~ | ~ \forall i, ~ |x_i| = 1 \}$ et le groupe $G \leqslant SO_3(\mathbb{R}^3)$ préservant globalement $\mathcal{C}$.
		
		L'opération est fidèle, car si $g$ est une application fixant trois points distincts, alors elle possède trois vecteurs propres linéairement indépendants de valeur propre 1, elle est donc diagonalisable et est égale à l'identité.
		
		Elle est transitive car tout sommet peut être envoyé sur un autre.
		
		Elle n'est pas libre car la rotation autour d'une "diagonale" fixe les sommets par laquelle elle passe.
	\end{enumerate}
\end{myexmpl}

\subsection{Aspects numériques}

\begin{myproposition}
	\begin{enumerate}
		\item Étant donné une opération de $G$ sur $X$, il existe une application bijective
		
		$$\func{\varphi}{G/Stab(x)}{G \cdot x}{g Stab(x)}{g \cdot x}$$
		
		\item Si $G$ et $X$ sont finis, alors $$|G \cdot x| = \frac{|G|}{|Stab(x)|}$$
		
		\item Si $G$ et $X$ sont finis, on choisit pour chaque orbite $\omega$ un élément $x_\omega \in \omega$ et on obtient :
		
		$$|X| = \sum_{\omega \in G \backslash X} \frac{|G|}{Stab_G(x_\omega)}$$
	\end{enumerate}
	
	Les deux égalités s'appellent \textit{équations aux classes}.
\end{myproposition}

\begin{myproof}
	Soit $x \in G$, on considère l'action de $Stab_G(x)$ sur $G$ par translation à droite, et $\sim$ la relation d'équivalence sur $G$ associée à celle-ci.
	
	On définit : $$\func{f}{G}{G \cdot x}{g}{g \cdot x}$$
	
	qui est constante sur les classes d'équivalences $G/\sim$, donc il existe $\funcshort{\varphi}{G/Stab_G(x)}{G \cdot x}$ telle que $\forall g \in G, ~ \varphi(gStab_G(x))=g \cdot x$
	
	Montrons que $\varphi$ est bijective.
	
	\begin{proofpart}{$\varphi$ est surjective}
		Soit $y \in G \cdot x$, il existe $g \in G$ tel que $y = g \cdot x = f(g) = \varphi(g Stab_G(x))$.
	\end{proofpart}
	
	\begin{proofpart}
		Soient $\alpha, \beta \in G/Stab_G(x)$ tels que $\varphi(\alpha) = \varphi(\beta)$.
		
		Il existe $g, h \in G$ tels que $\alpha = g Stab_G(x)$ et $\beta = h Stab_G(x)$.
		
		Alors $g \cdot x=f(g)=\varphi(\alpha)=\varphi(\beta)=f(h)=h \cdot x$, d'où $h^{-1} \star g \in Stab_G(x)$ ainsi $gStab_G(x)=hStab_G(x)$, on obtient alors $\alpha = \beta$.
	\end{proofpart}
	
	\cqfd
\end{myproof}

\begin{myexer}
	Pour chaque classe $\omega \in G/Stab_G(x)$ on pose $g_\omega \in \omega$, alors pour tout $g \in G$ il existe un unique couple $(\omega, h) \in G/Stab_G(x) \times Stab_G(x)$ tel que $g = g_\omega \star h$
\end{myexer}


\part{Groupes symétriques}
\part{Sous-groupes distingués et groupes quotient}
\part{Théorème de Sylow}

\part{Solutions des exercices}

\paragraph{Solution de l'exercice 1}

Commençons par montrer pour tout $n > 0$, $( g^n )^{-1} = g^{-n}$ :

$$\left( g^n \right)^{-1} = (g * g^{n-1})^{-1} = ((g^{n-1})^{-1}*g^{-1})^{-1}$$

$$\left( g^n \right)^{-1} = ((g^{n-2})^{-1}*g^{-1}*g^{-1})^{-1}$$

$$\cdots$$

$$\left( g^n \right)^{-1} = \underbrace{g^{-1}*g^{-1} ... g^{-1}}_{n \text{ fois}} = (g^{-1})^n = g^{-n}$$

Pour tout $m, n \in \mathbb{Z}$, on distingue plusieurs cas :
\begin{itemize}
	\item $m = 0$ ou $n = 0$ \checkmark
	\item $m, n > 0$ : \checkmark
	\item $m > 0, n < 0$ avec $m + n < 0$ : $$g^m * g^n = g^m * \left(g^{-1}\right)^{|n|} = g^m*\left(g^{-1}\right)^m*\left(g^{-1}\right)^{|n| - m} = e * \left(g^{-1}\right)^{|n|-m}=\left(g^{-1}\right)^{-n-m}=g^{m+n}$$
	\item $m, n < 0$ : $$g^{m+n}=\left(g^{-1}\right)^{|m|+|n|}=\left(g^{-1}\right)^{|m|}*\left(g^{-1}\right)^{|n|}=g^m*g^n$$
	\item les autres cas se démontrent de la même façon
\end{itemize}

\paragraph{Solution de l'exercice 2}

Supposons par l'absurde que $\left(\mathbb{Z}^G, \otimes\right)$ est un groupe :

\subparagraph{Stabilité de l'opération :} \checkmark

\subparagraph{Élément neutre :} On cherche $\funcshort{\epsilon}{G}{\mathbb{Z}}$ tel que
$$\forall f \in \mathbb{Z}^G, ~ \forall g \in G, ~ \sum_{h \in G}\epsilon(h)f(h^{-1}*g)=\sum_{h \in G}f(h)\epsilon(h^{-1}*g)=f(g)$$

Pour $f$ valant $1$ sur $G$ on a
$$\sum_{h \in G}\epsilon(h)=\sum_{h \in G}\epsilon(h^{-1}*g)=1$$

Vérifions que si $\epsilon$ est définie par $\epsilon(g) = \left\{
\begin{array}{l}
	1, \text{ si } g = e \\
	0, \text{ sinon}
\end{array}
\right.$, alors elle est neutre pour $\otimes$ :

$$\sum_{h \in G}\underbrace{\epsilon(h)}_{1 \text{ \textit{ssi} } h = e}f(h^{-1}*g) = f(e^{-1}*g) = f(g)$$

$$\sum_{h \in G}f(h)\underbrace{\epsilon(h^{-1}*g)}_{1 \text{ ssi } h = g}=f(g)$$

\checked

\subparagraph{Existence d'un inverse :}
Soit $\funcshort{f}{G}{\mathbb{Z}}$, il existe $\funcshort{\varphi}{G}{\mathbb{Z}}$ telle que $f \otimes \varphi = \varphi \otimes f = \epsilon$

$$\forall g \neq e, ~ \sum_{h \in G}\varphi(h)f(h^{-1}*g)=\sum_{h \in G}f(h)\varphi(h^{-1}*g)=0$$

et

$$\sum_{h \in G}\varphi(h)f(h^{-1})=\sum_{h \in G}f(h)\varphi(h^{-1})=1$$

la deuxième égalité est impossible lorsque $f$ est la fonction nulle, $\left(\mathbb{Z}^G, \otimes\right)$ n'est donc pas un groupe.

\paragraph{Solution de l'exercice 3}
Soit $K$ un sous-groupe vérifiant les propriétés suivantes :

$$
	\begin{array}{lc}
		(1) & \forall H \leqslant G, ~ A \subseteq H \Longrightarrow K \subseteq H \\
		(2) & A \subseteq K \leqslant G
	\end{array}
$$

On rappelle que 

$$
	\begin{array}{lc}
		(3) & \forall H \leqslant G, ~ A \subseteq H \Longrightarrow \langle A \rangle \subseteq H \\
		(4) & A \subseteq \langle A \rangle \leqslant G
	\end{array}
$$

$A \subseteq K$ alors d'après (3) $\langle A \rangle \subseteq K$ et $A \subseteq \langle A \rangle$ alors d'après (1) $K \subseteq \langle A \rangle$

\paragraph{Solution de l'exercice 4}
On pose $A = \{g^n ~ | ~ n \geqslant 0\}$.

$g \in A$ donc $\langle g \rangle \subseteq A$, de plus $g \in \langle g \rangle$ alors par récurrence $\forall n \geqslant 0, ~ g^n \in \langle g \rangle$, d'où $A \subset \langle g \rangle$.

\paragraph{Solution de l'exercice 6}

Soit $d > 0$ et $g \in G$, montrons l'équivalence entre les deux propositions suivantes

$$
\begin{array}{ll}
	(i) & d \text{ est l'odre de } g \\
	(ii) & g^d=e \text{ et } \forall k | d, ~ (k < d \Longrightarrow g^k \neq e)
\end{array}
$$

\noindent
\begin{proofpart}{$(i) \Longrightarrow (ii)$}

	$d$ étant l'ordre de $g$, on a $g^d=e$, et par minimalité de $d$ on a pour tout $k < d$, $g^k \neq e$ (en particulier pour tout diviseur strict de $d$).
\end{proofpart}

\begin{proofpart}{$(ii) \Longrightarrow (i)$}

	$d$ vérifie :
	\begin{enumerate}
		\item $g^d=e$
		\item $\forall k < d, ~ (k | d \Longrightarrow g^k \neq e)$
	\end{enumerate}
	
	On a que $d \geqslant ord(g)$, par minimalité de $ord(g)$.
	
	Supposons maintenant que $d \neq ord(g)$, c'est à dire que $d > ord(g)$, l'ordre de $g$ divise nécessairement $d$, d'où l'existence d'un entier $n > 1$ tel que $d= n \cdot ord(g)$.
	
	$ord(g)$ est donc un diviseur strict de $d$ ! $d$ est ainsi égal à l'ordre de $g$, sinon on aurait d'après (2) $g^{ord(g)} \neq e$
\end{proofpart}

\paragraph{Solution de l'exercice 7}

Soit $\func{f}{\mathbb{U}_n}{\mathbb{U}_n}{x}{x^d}$

\subparagraph{Noyau}

$\ker f = \{x \in \mathbb{U}_n ~ | ~ x^d = e \} = \mathbb{U}_d$

$Im(f)=\{x^d ~|~ x \in \mathbb{U}_n\}=\mathbb{U}_{\frac{n}{n \land d}}$

\paragraph{Solution de l'exercice 8}

On considère l'action des applications linéaires inversibles de $E$ sur les formes quadratiques de $E$ définie par $g \cdot q = q \circ g^{-1}$, montrons que c'est une action de groupe.

\begin{proofpart}{Composition}

	Soient $f, g \in GL(E)$ et une forme quadratique $q$ :
	
	$f\cdot(g \cdot q)=(g \cdot q) \circ f^{-1}=q\circ g^{-1}\circ f^{-1}=q \circ (f \circ g)^{-1}=(f \circ g) \cdot q$
\end{proofpart}

\begin{proofpart}{Élément neutre}

	$id \cdot q = q \circ id^{-1} = q$
\end{proofpart}

\paragraph{\boldmath $C_m \times C_n \cong C_{mn}$} Soient $m$ et $n$ premiers entre eux.

On considère l'application
$$\func{\varphi}{C_m \times C_n}{C_{mn}}{(g,h)}{gh}$$

Pour tout $(g, h), (g', h') \in C_m \times C_n$, on a :  $$\varphi((g,h)(g',h')=\varphi(gg',hh')=gg'hh'=(gh)(g'h')=\varphi(g,h)\varphi(g',h')$$

Donc $\varphi$ est un morphisme, de plus si elle est injective alors elle sera bijective car $|C_{mn}| = |C_m \times C_n|$.

Soit $(g,h) \in C_m \times C_n$ tel que $\varphi(g, h) = gh = e$

$g \in C_m$ donc l'ordre de $g$ divise $m$, de même l'ordre de $h^{-1}$ divise $n$. On a donc que l'ordre de $g=h^{-1}$ divise $m$ et $n$, or $m$ et $n$ sont premiers entre eux, alors l'ordre de $g$ divise $m \land n = 1$.

Ainsi $g = h = e$, $\varphi$ est donc injective et donc un isomorphisme.

\paragraph{Solution de l'exercice 10}

Montrons que pour tout $g \in G$ il existe un unique couple $(\omega, h) \in G / \text{Stab}(x) \times \text{Stab}(x)$ tel que $g_\omega \star h = g$, c'est-à-dire qu'il existe une bijection entre les ensembles $G$ et $G/\text{Stab}(x) \times \text{Stab}(x)$.

On pose $$\func{\varphi}{G/\text{Stab}(x) \times \text{Stab}(x)}{G}{(\omega, h)}{g_\omega \star h}$$

Pour tout $g \in G$, $g \in \omega = \text{Stab}(x)$ et il existe un certain $g_\omega \in \omega$ tel que $g_\omega \text{Stab}(x) = g \text{Stab}(x)$, c'est-à-dire qu'il existe un $h \in \text{Stab}(x)$ tel que $g_\omega \star h = g$, d'où la surjectivité de $\varphi$.

$\varphi$ est bijective car $\left|G / \text{Stab}(x) \times \text{Stab}(x)\right| = \frac{|G|}{\text{Stab}(x)} \cdot |\text{Stab}(x)| = |G|$.

\end{document}