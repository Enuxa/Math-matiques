\documentclass[]{article}
\usepackage[utf8]{inputenc}
\usepackage{pdfpages}
\usepackage{amsmath}
\usepackage{amssymb}
\usepackage{graphicx}
\usepackage{geometry}
\usepackage{enumitem}
\usepackage{amsthm}
\usepackage{stmaryrd}

\geometry{hmargin=2cm}

\title{Logique II}
\author{Paul Rozière - Notes prises par Pierre Gervais}

% Environnement type théorème
\newtheorem{mythm}{Théorème}
\newtheorem{myproposition}{Proposition}
\newtheorem{myproperty}{Propriété}
\newtheorem{mylemma}{Lemme}

% Environnement type texte
\theoremstyle{remark}
\newtheorem{mynot}{Notation}
\newtheorem{myrem}{Remarque}
\newtheorem{myexer}{Exercice}
\newtheorem{myproof}{Preuve}
\newtheorem{myexmpl}{Exemple}
\newtheorem{myrapl}{Rappel}
\newtheorem{mycor}{Corollaire}

% Environnement de définition
\theoremstyle{definition}
\newtheorem{mydef}{Définition}
\newtheorem{myquestion}{Question}

\setlist[itemize]{label=-}

% Carré de fin de preuve
\newcommand{\cqfd}{
	\hfill$\square$
}

% Définition de fonction
\newcommand{\func}[5]{
#1 ~ : ~ \left\{ \begin{array}{lcl}
	#2 & \longrightarrow & #3 \\
	#4 & \longmapsto & #5
\end{array}
\right.
}

\newcommand{\fun}[3]{
#1 ~ : ~ #2 \longrightarrow #3
}

\newcommand{\funcinline}[5]{
	#1 \, : \, #2 \longrightarrow #3, ~ #4 \longmapsto #5
}

\newcommand{\funcshort}[3]{
	#1 \, : \, #2 \longrightarrow #3
}

\newcommand{\anonfunc}[4]{
	\left\{ \begin{array}{lcl}
		#1 & \longrightarrow & #2 \\
		#3 & \longmapsto & #4
	\end{array}
	\right.
}

\newcommand{\DS}{\displaystyle}

\begin{document}

\maketitle

https://www.irif.fr/$\sim$roziere/logiqueL3MIS2/

\tableofcontents

\newpage

\section{Axiomatisation de l'arithmétique}

On cherche à décrire la notion d'entier naturel. (fin 19e siècle, Dedekind et Peano).

\subsection{Définition inductive}

On suppose se placer dans un contexte où 0 et la fonctions successeur $s$ sont définis avec $s(x) \neq 0$ pour tout $x$ et $s$ injectif.

Un ensemble $A$ possède la propriété de clôture $cl(A)$ si $0 \in A$ et $\forall x \in A, ~ s(x) \in A$.

$\mathbb{N}$ peut être défini comme le plus petit ensemble $A$ vérifiant $cl(A)$, c'est-à-dire $$\mathbb{N} = \bigcap_{A ~ : ~ cl(A)} A$$

Pour que cela fonctionne, on doit
\begin{enumerate}
	\item avoir au moins un $A$ vérifiant $cl(A)$
	\item pour toute famille $(A_i)_{i \in I}$ vérifiant chacun $cl(A_i)$, on a $\DS cl\left(\bigcap_{i \in I} A_i\right)$
\end{enumerate}

La propriété (1) est admise, pour la propriété 2 :
\begin{itemize}
	\item Pour tout $i \in I$, $cl(A_i)$, alors $0 \in Au_i$ et si $x \in \bigcap_i A_i$ alors $\forall i \in I, x \in A_i$ et $s(x) \in A_i$ donc $s(x) \in \bigcap_i A_i$ et $cl\left(\bigcap_i A_i\right)$
\end{itemize}

\begin{myproperty} Propriété de récurrence
	
	Pour toute propriété "bien définie" sur $\mathbb{N}$ $P$ vérifiant :
	\begin{itemize}
		\item $P(0)$
		\item $\forall n \in \mathbb{N}, ~ (P(n) \Longrightarrow P(n+1))$
	\end{itemize}
	
	on obtient, en posant $A = \{n \in \mathbb{N} ~ | ~ P(n)\}$ que $0 \in A$ et $\forall n \in A, ~ n+1 \in A$ donc $A = \mathbb{N}$
\end{myproperty}

\subsection{Approche axiomatique (définition implicite)}

\subsubsection{Axiomes de Peano}

Soient $\mathbb{N}, 0, s$ où $0 \in \mathbb{N}$ et $\funcshort{s}{\mathbb{N}}{\mathbb{N}}$, les axiomes de Peano sont les suivants :
\begin{enumerate}
	\item Pour tout $x \in \mathbb{N}$, $s (x) \neq 0$
	\item $s$ est injectif
	\item propriété de récurrence : $\mathbb{N}$ est l'unique partie de $\mathbb{N}$ qui contient 0 et le successeur de tout ses éléments
\end{enumerate}

\subsubsection{Définition par récurrence}

\begin{myexmpl}
	L'addition : $x + 0 = x$ et $x + s(y) = s (x + y)$
\end{myexmpl}

\begin{mythm} Définition par récurrence (énoncé par Dedekind)
	
	Soit $E$ un ensemble, $a \in E$ et $\funcshort{h}{E}{E}$, il existe une et une seule fonction $\funcshort{f}{\mathbb{N}}{E}$ vérifiant
	$$
	(*) ~\left\{
	\begin{matrix}{c}
		f(0) = a \\
		\forall n \in \mathbb{N}, ~ f(s(n)) = h(f(n))
	\end{matrix}
	\right.
	$$
\end{mythm}

On a besoin que $\mathbb{N}$ soit "librement engendré" à partir de 0

\begin{myproof}
	\leavevmode
	\paragraph{Unicité}
	Par récurrence \checkmark
	\paragraph{Existence}
	on va donner une définition inductive du graphe de $f$ (qui est un sous ensemble de $\mathbb{N} \times E$).
	Soit $A \subseteq \mathbb{N} \times E$, on définit une nouvelle propriété de cloture pour $A$, notée $cl(A)$ :
	\begin{itemize}
		\item $(0, a) \in A$
		\item si $(n, x) \in A$ alors $(s(n), h(x)) \in A$
	\end{itemize}
	
	Montrons $G := \bigcap_{A ~ : ~ cl(A)} A$ est le graphe d'une fonction $f$ vérifiant $(*)$
	
	\begin{itemize}
		\item Pour tout $n \in \mathbb{N}$ il existe $y \in E$ tel que $(n, y) \in G$ (partout définie)
		\item Pour tout $n \in \mathbb{N}$ $\forall x, y \in E$ $(n, x) \in G$ et $(n ,y) \in G \Longrightarrow x = y$ (unicité de l'image)
	\end{itemize}
	
	Premier point par récurrence sur $n$ :
	\begin{itemize}
		\item $n=0$ : $(0, a) \in G$ car $cl(G)$
		\item $n > 0$ : supposons $(n, x) \in G$ alors ar $cl(G)$ : $(s(n), h(x)) \in G$
	\end{itemize}
	
	Second point par récurrence sur $n$ :
	\begin{itemize}
		\item $(0, x) \in G$ et $(0, y) \in G$, $x \neq a$ ou $y \neq a$ (disons $x \neq a$)
		$G' := G \setminus \{(0, x)\}$ vérifie la propriété de cloture $cl(G)$, $(0, a) \in G'$.
		
		Si $(n, y) \in G'$, $(s(n), h(y)) \in G'$ car $s(n) \neq 0$, on aurait $G \subseteq G'$, contradiction.
		
		\item suppososns $(n, x) \in G \text{ et } (x, y) \in G \Longrightarrow x = y$
		
		soit $f_n$ tel que $(n, f_n) \in G$ (on a vu l'existence) supposons $(s(n), x) \in G$ et $(s(n), y) \in G$
		
		supposons $x \neq h(f_n)$, si $m = n$ alors $(n, f_n) \in G$ donc $\in G'$ car $n \neq s(n)$
		
		$x \neq h(z)$, donc $(n, h(f_n)) \in G'$, on a bien l'unicité.
	\end{itemize}
	\cqfd
\end{myproof}

\begin{mycor} Définition par récurrence avec paramètres
	
	Soit $A$ un ensemble et $E$ un ensemble non-vide, deux fonctions $\funcshort{g}{A}{E}$ et $\funcshort{h}{(E \times A)}{E}$ alors il existe une unique fonction $\funcshort{f}{\mathbb{N} \times A}{E}$ vérifiant
	\begin{itemize}
		\item $f(0, y) = g(y)$
		\item $f(s(n), y) = h(f(n, y), y)$
	\end{itemize}
\end{mycor}

\end{document}
